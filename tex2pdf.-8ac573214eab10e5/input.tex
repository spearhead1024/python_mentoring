\PassOptionsToPackage{unicode=true}{hyperref} % options for packages loaded elsewhere
\PassOptionsToPackage{hyphens}{url}
%
\documentclass[]{article}
\usepackage{lmodern}
\usepackage{amssymb,amsmath}
\usepackage{ifxetex,ifluatex}
\usepackage{fixltx2e} % provides \textsubscript
\ifnum 0\ifxetex 1\fi\ifluatex 1\fi=0 % if pdftex
  \usepackage[T1]{fontenc}
  \usepackage[utf8]{inputenc}
  \usepackage{textcomp} % provides euro and other symbols
\else % if luatex or xelatex
  \usepackage{unicode-math}
  \defaultfontfeatures{Ligatures=TeX,Scale=MatchLowercase}
\fi
% use upquote if available, for straight quotes in verbatim environments
\IfFileExists{upquote.sty}{\usepackage{upquote}}{}
% use microtype if available
\IfFileExists{microtype.sty}{%
\usepackage[]{microtype}
\UseMicrotypeSet[protrusion]{basicmath} % disable protrusion for tt fonts
}{}
\IfFileExists{parskip.sty}{%
\usepackage{parskip}
}{% else
\setlength{\parindent}{0pt}
\setlength{\parskip}{6pt plus 2pt minus 1pt}
}
\usepackage{hyperref}
\hypersetup{
            pdfborder={0 0 0},
            breaklinks=true}
\urlstyle{same}  % don't use monospace font for urls
\setlength{\emergencystretch}{3em}  % prevent overfull lines
\providecommand{\tightlist}{%
  \setlength{\itemsep}{0pt}\setlength{\parskip}{0pt}}
\setcounter{secnumdepth}{0}
% Redefines (sub)paragraphs to behave more like sections
\ifx\paragraph\undefined\else
\let\oldparagraph\paragraph
\renewcommand{\paragraph}[1]{\oldparagraph{#1}\mbox{}}
\fi
\ifx\subparagraph\undefined\else
\let\oldsubparagraph\subparagraph
\renewcommand{\subparagraph}[1]{\oldsubparagraph{#1}\mbox{}}
\fi

% set default figure placement to htbp
\makeatletter
\def\fps@figure{htbp}
\makeatother


\date{}

\begin{document}

\hypertarget{chapter-1}{%
\section{Chapter 1}\label{chapter-1}}

\hypertarget{howler}{%
\section{Howler}\label{howler}}

Write a Python program \texttt{howler.py} that will uppercase all the
text from the command line or from a file.

\begin{verbatim}
$ ./howler.py
usage: howler.py [-h] [-o str] STR
howler.py: error: the following arguments are required: STR
$ ./howler.py -h
usage: howler.py [-h] [-o str] STR

Howler (upper-case input)

positional arguments:
  STR                   Input string or file

optional arguments:
  -h, --help            show this help message and exit
  -o str, --outfile str
                        Output filename (default: )
\end{verbatim}

\hypertarget{skills}{%
\section{Skills}\label{skills}}

\begin{itemize}
\tightlist
\item
  Reading text from command line or a file
\item
  Transforming text
\item
  Write to a file or STDOUT
\end{itemize}

\pagebreak

\hypertarget{solution}{%
\section{Solution}\label{solution}}

\begin{verbatim}
#!/usr/bin/env python3
"""
Author : kyclark
Date   : 2019-05-17
Purpose: Howler
"""

import argparse
import os
import sys


# --------------------------------------------------
def get_args():
    """get command-line arguments"""
    parser = argparse.ArgumentParser(
        description='Howler (upper-case input)',
        formatter_class=argparse.ArgumentDefaultsHelpFormatter)

    parser.add_argument('text', metavar='STR', help='Input string or file')

    parser.add_argument(
        '-o',
        '--outfile',
        help='Output filename',
        metavar='str',
        type=str,
        default='')

    return parser.parse_args()


# --------------------------------------------------
def warn(msg):
    """Print a message to STDERR"""
    print(msg, file=sys.stderr)


# --------------------------------------------------
def die(msg='Something bad happened'):
    """warn() and exit with error"""
    warn(msg)
    sys.exit(1)


# --------------------------------------------------
def main():
    """Make a jazz noise here"""
    args = get_args()
    text = args.text
    out_file = args.outfile

    if os.path.isfile(text):
        text = open(text).read()

    out_fh = open(out_file, 'wt') if out_file else sys.stdout
    out_fh.write(text.upper() + '\n')


# --------------------------------------------------
if __name__ == '__main__':
    main()
\end{verbatim}

\pagebreak

\hypertarget{chapter-2}{%
\section{Chapter 2}\label{chapter-2}}

\hypertarget{jump-the-five}{%
\section{Jump the Five}\label{jump-the-five}}

Write a program called \texttt{jump.py} that will encode any number
using ``jump-the-five'' algorithm that selects as a replacement for a
given number the number that is opposite the number on a US telephone
pad if you jump over the 5. The numbers 5and 9 will exchange with each
other. So, ``1'' jumps the 5 to become ``9,'' ``6'' jumps the 5 to
become ``4,'' ``5'' becomes ``0,'' etc.

\begin{verbatim}
   1  2  3
   4  5  6
   7  8  9
   #  0  *
\end{verbatim}

If given no arguments, print a usage statement.

\hypertarget{expected-behavior}{%
\section{Expected Behavior}\label{expected-behavior}}

\begin{verbatim}
$ ./jump.py
Usage: jump.py NUMBER
$ ./jump.py 555-1212
000-9898
$ ./jump.py 'Call 1-800-329-8044 today!'
Call 9-255-781-2566 today!
\end{verbatim}

\pagebreak

\hypertarget{solution-1}{%
\section{Solution}\label{solution-1}}

\begin{verbatim}
#!/usr/bin/env python3
"""
Author : Ken Youens-Clark <kyclark@gmail.com>
Date   : 2019-05-06
Purpose: Jump the Five
"""

import os
import sys


# --------------------------------------------------
def main():
    args = sys.argv[1:]

    if len(args) != 1:
        print('Usage: {} NUMBER'.format(os.path.basename(sys.argv[0])))
        sys.exit(1)

    num = args[0]
    jumper = {
        '1': '9',
        '2': '8',
        '3': '7',
        '4': '6',
        '5': '0',
        '6': '4',
        '7': '3',
        '8': '2',
        '9': '1',
        '0': '5'
    }

    for char in num:
        print(jumper[char] if char in jumper else char, end='')
    print()


# --------------------------------------------------
if __name__ == '__main__':
    main()
\end{verbatim}

\pagebreak

\hypertarget{chapter-3}{%
\section{Chapter 3}\label{chapter-3}}

\hypertarget{bottles-of-beer-song}{%
\section{Bottles of Beer Song}\label{bottles-of-beer-song}}

Write a Python program called \texttt{bottles.py} that takes a single
option \texttt{-n\textbar{}-\/-num\_bottles} which is an positive
integer (default 10) and prints the `` bottles of beer on the wall
song.'' If the \texttt{-n} argument is less than 1, die with ``N () must
be a positive integer''. The program should also respond to
\texttt{-h\textbar{}-\/-help} with a usage statement.

I'd encourage you to think about the program as a formal algorithm. Read
the introduction to Jeff Erickson's book \emph{Algorithms} available
here:

\begin{itemize}
\tightlist
\item
  http://jeffe.cs.illinois.edu/teaching/algorithms/\#book
\item
  http://jeffe.cs.illinois.edu/teaching/algorithms/book/00-intro.pdf
\end{itemize}

You are going to need to count down, so you'll need to consider how to
do that. First, let's examine a list and see how it can be sorted and
reversed. We've already used the \texttt{sorted} \emph{function}, but we
haven't really talked about the \texttt{list} class's \texttt{sort}
\emph{method}. Note that the former does not mutate the list itself:

\begin{verbatim}
>>> a = ['foo', 'bar', 'baz']
>>> sorted(a)
['bar', 'baz', 'foo']
>>> a
['foo', 'bar', 'baz']
\end{verbatim}

But the \texttt{sort} method does:

\begin{verbatim}
>>> a.sort()
>>> a
['bar', 'baz', 'foo']
\end{verbatim}

Also, note what is returned by \texttt{sort}:

\begin{verbatim}
>>> type(a.sort())
<type 'NoneType'>
\end{verbatim}

So if you did this, you'd destroy your data:

\begin{verbatim}
>>> a = a.sort()
>>> a
\end{verbatim}

As with \texttt{sort}/\texttt{sorted}, so it goes with
\texttt{reverse}/\texttt{reversed}. The past participle version
\emph{returns a new copy of the data without affecting the original} and
is therefore the safest bet to use:

\begin{verbatim}
>>> a = ['foo', 'bar', 'baz']
>>> a
['foo', 'bar', 'baz']
>>> reversed(a)
<listreverseiterator object at 0x10f0d61d0>
>>> list(reversed(a))
['baz', 'bar', 'foo']
>>> a
['foo', 'bar', 'baz']
\end{verbatim}

Compare with:

\begin{verbatim}
>>> a.reverse()
>>> a
['baz', 'bar', 'foo']
\end{verbatim}

Given that and your knowledge of how \texttt{range} works, can you
figure out how to count down, say, from 10 to 1?

\hypertarget{expected-behavior-1}{%
\section{Expected Behavior}\label{expected-behavior-1}}

\begin{verbatim}
$ ./bottles.py -h
usage: bottles.py [-h] [-n INT]

Bottles of beer song

optional arguments:
  -h, --help            show this help message and exit
  -n INT, --num_bottles INT
$ ./bottles.py --help
usage: bottles.py [-h] [-n INT]

Bottles of beer song

optional arguments:
  -h, --help            show this help message and exit
  -n INT, --num_bottles INT
                        How many bottles (default: 10)
$ ./bottles.py -n 1
1 bottle of beer on the wall,
1 bottle of beer,
Take one down, pass it around,
0 bottles of beer on the wall!

$ ./bottles.py
10 bottles of beer on the wall,
10 bottles of beer,
Take one down, pass it around,
9 bottles of beer on the wall!

9 bottles of beer on the wall,
9 bottles of beer,
Take one down, pass it around,
8 bottles of beer on the wall!

8 bottles of beer on the wall,
8 bottles of beer,
Take one down, pass it around,
7 bottles of beer on the wall!

7 bottles of beer on the wall,
7 bottles of beer,
Take one down, pass it around,
6 bottles of beer on the wall!

6 bottles of beer on the wall,
6 bottles of beer,
Take one down, pass it around,
5 bottles of beer on the wall!

5 bottles of beer on the wall,
5 bottles of beer,
Take one down, pass it around,
4 bottles of beer on the wall!

4 bottles of beer on the wall,
4 bottles of beer,
Take one down, pass it around,
3 bottles of beer on the wall!

3 bottles of beer on the wall,
3 bottles of beer,
Take one down, pass it around,
2 bottles of beer on the wall!

2 bottles of beer on the wall,
2 bottles of beer,
Take one down, pass it around,
1 bottle of beer on the wall!

1 bottle of beer on the wall,
1 bottle of beer,
Take one down, pass it around,
0 bottles of beer on the wall!
\end{verbatim}

\pagebreak

\hypertarget{solution-2}{%
\section{Solution}\label{solution-2}}

\begin{verbatim}
#!/usr/bin/env python3
"""
Author : Ken Youens-Clark <kyclark@gmail.com>
Date   : 2019-03-01
Purpose: Bottles of beer
"""

import argparse
import sys


# --------------------------------------------------
def get_args():
    """get command-line arguments"""
    parser = argparse.ArgumentParser(
        description='Bottles of beer song',
        formatter_class=argparse.ArgumentDefaultsHelpFormatter)

    parser.add_argument(
        '-n',
        '--num_bottles',
        metavar='INT',
        type=int,
        default=10,
        help='How many bottles')

    return parser.parse_args()


# --------------------------------------------------
def warn(msg):
    """Print a message to STDERR"""
    print(msg, file=sys.stderr)


# --------------------------------------------------
def die(msg='Something bad happened'):
    """warn() and exit with error"""
    warn(msg)
    sys.exit(1)


# --------------------------------------------------
def main():
    """Make a jazz noise here"""
    args = get_args()
    num_bottles = args.num_bottles

    if num_bottles < 1:
        die('N ({}) must be a positive integer'.format(num_bottles))

    line1 = '{} bottle{} of beer on the wall'
    line2 = '{} bottle{} of beer'
    line3 = 'Take one down, pass it around'
    tmpl = ',\n'.join([line1, line2, line3, line1 + '!'])

    for n in reversed(range(1, num_bottles + 1)):
        s1 = '' if n == 1 else 's'
        s2 = '' if n - 1 == 1 else 's'
        print(tmpl.format(n, s1, n, s1, n - 1, s2))
        if n > 1: print()


# --------------------------------------------------
if __name__ == '__main__':
    main()
\end{verbatim}

\pagebreak

\hypertarget{chapter-4}{%
\section{Chapter 4}\label{chapter-4}}

\hypertarget{picnic}{%
\section{Picnic}\label{picnic}}

Write a Python program called \texttt{picnic.py} that accepts one or
more positional arguments as the items to bring on a picnic. In
response, print ``You are bringing \ldots{}'' where ``\ldots{}'' should
be replaced according to the number of items where:

\begin{enumerate}
\def\labelenumi{\arabic{enumi}.}
\tightlist
\item
  If one item, just state, e.g., if ``chips'' then ``You are bringing
  chips.''
\item
  If two items, put ``and'' in between, e.g., if ``chips soda'' then
  ``You are bringing chips and soda.''
\item
  If three or more items, place commas between all the items INCLUDING
  BEFORE THE FINAL ``and'' BECAUSE WE USE THE OXFORD COMMA, e.g., if
  ``chips soda cupcakes'' then ``You are bringing chips, soda, and
  cupcakes.''
\end{enumerate}

\begin{verbatim}
$ ./picnic.py
usage: picnic.py [-h] str [str ...]
picnic.py: error: the following arguments are required: str
$ ./picnic.py -h
usage: picnic.py [-h] str [str ...]

Picnic game

positional arguments:
  str         Item(s) to bring

optional arguments:
  -h, --help  show this help message and exit
$ ./picnic.py chips
You are bringing chips.
$ ./picnic.py "potato chips" salad
You are bringing potato chips and salad.
$ ./picnic.py "potato chips" salad soda cupcakes
You are bringing potato chips, salad, soda, and cupcakes.
\end{verbatim}

\pagebreak

\hypertarget{solution-3}{%
\section{Solution}\label{solution-3}}

\begin{verbatim}
#!/usr/bin/env python3
"""
Author : Ken Youens-Clark <kyclark@gmail.com>
Date   : 2019-05-24
Purpose: Picnic game
"""

import argparse


# --------------------------------------------------
def get_args():
    """Get command-line arguments"""

    parser = argparse.ArgumentParser(
        description='Picnic game',
        formatter_class=argparse.ArgumentDefaultsHelpFormatter)

    parser.add_argument('item',
                        metavar='str',
                        nargs='+',
                        help='Item(s) to bring')

    return parser.parse_args()


# --------------------------------------------------
def main():
    """Make a jazz noise here"""

    args = get_args()
    items = args.item
    num = len(items)

    bringing = items[0] if num == 1 else ' and '.join(
        items) if num == 2 else ', '.join(items[:-1] + ['and ' + items[-1]])

    print('You are bringing {}.'.format(bringing))


# --------------------------------------------------
if __name__ == '__main__':
    main()
\end{verbatim}

\pagebreak

\hypertarget{chapter-5}{%
\section{Chapter 5}\label{chapter-5}}

\hypertarget{apples-and-bananas}{%
\section{Apples and Bananas}\label{apples-and-bananas}}

Perhaps you remember the children's song ``Apples and Bananas''?

\begin{verbatim}
I like to eat, eat, eat apples and bananas
I like to eat, eat, eat apples and bananas

I like to ate, ate, ate ay-ples and ba-nay-nays
I like to ate, ate, ate ay-ples and ba-nay-nays

I like to eat, eat, eat ee-ples and bee-nee-nees
I like to eat, eat, eat ee-ples and bee-nee-nees
\end{verbatim}

Write a Python program called \texttt{apples.py} that will turn all the
vowels in some given text in a single positional argument into just one
\texttt{-v\textbar{}-\/-vowel} (default ``a'') like this song. It should
complain if the \texttt{-\/-vowel} argument isn't a single, lowercase
vowel (hint, see \texttt{choices} in the \texttt{argparse}
documentation). If the given text argument is a file, read the text from
the file. Replace all vowels with the given vowel, both lower- and
uppercase.

\begin{verbatim}
$ ./apples.py
usage: apples.py [-h] [-v str] str
apples.py: error: the following arguments are required: str
$ ./apples.py -h
usage: apples.py [-h] [-v str] str

Apples and bananas

positional arguments:
  str                  Input text or file

optional arguments:
  -h, --help           show this help message and exit
  -v str, --vowel str  The only vowel allowed (default: a)
$ ./apples.py -v x foo
usage: apples.py [-h] [-v str] str
apples.py: error: argument -v/--vowel: invalid choice: 'x' (choose from 'a', 'e', 'i', 'o', 'u')
$ ./apples.py foo
faa
$ ./apples.py ../inputs/fox.txt
Tha qaack brawn fax jamps avar tha lazy dag.
\end{verbatim}

\pagebreak

\hypertarget{solution-4}{%
\section{Solution}\label{solution-4}}

\begin{verbatim}
#!/usr/bin/env python3
"""
Author : Ken Youens-Clark <kyclark@gmail.com>
Date   : 2019-05-18
Purpose: Apples and bananas
"""

import argparse
import os
import re
import sys


# --------------------------------------------------
def get_args():
    """get command-line arguments"""
    parser = argparse.ArgumentParser(
        description='Apples and bananas',
        formatter_class=argparse.ArgumentDefaultsHelpFormatter)

    parser.add_argument('text', metavar='str', help='Input text or file')

    parser.add_argument('-v',
                        '--vowel',
                        help='The vowel(s) allowed',
                        metavar='str',
                        type=str,
                        default='a',
                        choices=list('aeiou'))

    return parser.parse_args()


# --------------------------------------------------
def warn(msg):
    """Print a message to STDERR"""
    print(msg, file=sys.stderr)


# --------------------------------------------------
def die(msg='Something bad happened'):
    """warn() and exit with error"""
    warn(msg)
    sys.exit(1)


# --------------------------------------------------
def main():
    """Make a jazz noise here"""
    args = get_args()
    text = args.text
    vowel = args.vowel

    if os.path.isfile(text):
        text = open(text).read()

    # Method 1: Iterate every character
    # new_text = []
    # for char in text:
    #     if char in 'aeiou':
    #         new_text.append(vowel)
    #     elif char in 'AEIOU':
    #         new_text.append(vowel.upper())
    #     else:
    #         new_text.append(char)
    # text = ''.join(new_text)

    # Method 2: str.replace
    # for v in 'aeiou':
    #     text = text.replace(v, vowel).replace(v.upper(), vowel.upper())

    # Method 3: Use a list comprehension
    # new_text = [
    #     vowel if c in 'aeiou' else vowel.upper() if c in 'AEIOU' else c
    #     for c in text
    # ]
    # text = ''.join(new_text)

    # Method 4: Define a function, use list comprehension
    def new_char(c):
        return vowel if c in 'aeiou' else vowel.upper() if c in 'AEIOU' else c

    # text = ''.join([new_char(c) for c in text])

    # Method 5: Use a `map` to iterate with a `lambda`
    # text = ''.join(
    #     map(
    #         lambda c: vowel if c in 'aeiou' else vowel.upper()
    #         if c in 'AEIOU' else c, text))

    # Method 6: `map` with the function
    text = ''.join(map(new_char, text))

    # Method 7: Regular expressions
    # text = re.sub('[aeiou]', vowel, text)
    # text = re.sub('[AEIOU]', vowel.upper(), text)

    print(text.rstrip())


# --------------------------------------------------
if __name__ == '__main__':
    main()
\end{verbatim}

\pagebreak

\hypertarget{chapter-6}{%
\section{Chapter 6}\label{chapter-6}}

\hypertarget{gashlycrumb}{%
\section{Gashlycrumb}\label{gashlycrumb}}

Write a Python program called \texttt{gashlycrumb.py} that takes a
letter of the alphabet as an argument and looks up the line in a
\texttt{-f\textbar{}-\/-file} argument (default
\texttt{gashlycrumb.txt}) and prints the line starting with that letter.

\begin{verbatim}
$ ./gashlycrumb.py
usage: gashlycrumb.py [-h] [-f str] str
gashlycrumb.py: error: the following arguments are required: str
$ ./gashlycrumb.py -h
usage: gashlycrumb.py [-h] [-f str] str

Gashlycrumb

positional arguments:
  str                 Letter

optional arguments:
  -h, --help          show this help message and exit
  -f str, --file str  Input file (default: gashlycrumb.txt)
$ ./gashlycrumb.py 3
I do not know "3".
$ ./gashlycrumb.py CH
"CH" is not 1 character.
$ ./gashlycrumb.py a
A is for Amy who fell down the stairs.
$ ./gashlycrumb.py z
Z is for Zillah who drank too much gin.
\end{verbatim}

\hypertarget{discussion}{%
\section{Discussion}\label{discussion}}

If you are not familiar with the work of Edward Gorey, please stop and
go read about him immediately,
e.g.~https://www.brainpickings.org/2011/01/19/edward-gorey-the-gashlycrumb-tinies/!

Write your own version of Gorey's text and pass in your version as the
\texttt{-\/-file}.

Write an interactive version that takes input directly from the user:

\begin{verbatim}
$ ./gashlycrumb_i.py
Please provide a letter [! to quit]: a
A is for Amy who fell down the stairs.
Please provide a letter [! to quit]: b
B is for Basil assaulted by bears.
Please provide a letter [! to quit]: !
Bye
\end{verbatim}

\pagebreak

\hypertarget{solution-5}{%
\section{Solution}\label{solution-5}}

\begin{verbatim}
#!/usr/bin/env python3
"""
Author : kyclark
Date   : 2019-05-17
Purpose: Gashlycrumb
"""

import argparse
import os
import sys


# --------------------------------------------------
def get_args():
    """get command-line arguments"""
    parser = argparse.ArgumentParser(
        description='Gashlycrumb',
        formatter_class=argparse.ArgumentDefaultsHelpFormatter)

    parser.add_argument('letter', help='Letter', metavar='str', type=str)

    parser.add_argument('-f',
                        '--file',
                        help='Input file',
                        metavar='str',
                        type=str,
                        default='gashlycrumb.txt')

    return parser.parse_args()


# --------------------------------------------------
def warn(msg):
    """Print a message to STDERR"""
    print(msg, file=sys.stderr)


# --------------------------------------------------
def die(msg='Something bad happened'):
    """warn() and exit with error"""
    warn(msg)
    sys.exit(1)


# --------------------------------------------------
def main():
    """Make a jazz noise here"""
    args = get_args()
    letter = args.letter.upper()
    file = args.file

    if not os.path.isfile(file):
        die('--file "{}" is not a file.'.format(file))

    if len(letter) != 1:
        die('"{}" is not 1 character.'.format(letter))

    lookup = {}
    for line in open(file):
        lookup[line[0]] = line.rstrip()

    if letter in lookup:
        print(lookup[letter])
    else:
        print('I do not know "{}".'.format(letter))


# --------------------------------------------------
if __name__ == '__main__':
    main()
\end{verbatim}

\pagebreak

\hypertarget{chapter-7}{%
\section{Chapter 7}\label{chapter-7}}

\hypertarget{movie-reader}{%
\section{Movie Reader}\label{movie-reader}}

Write a Python program called \texttt{movie\_reader.py} that takes a
single positional argument that is a bit of text or the name of an input
file. The output will be dynamic, so I cannot write a test for how the
program should behave, nor can I include a bit of text that shows you
how it should work. Your program should print the input text
character-by-character and then pause .5 seconds for ending punctuation
like \texttt{.}, \texttt{!} or \texttt{?}, .2 seconds for a pause like
\texttt{,} \texttt{:}, or \texttt{;}, and .05 seconds for anything else.

\begin{verbatim}
$ ./movie_reader.py
usage: movie_reader.py [-h] str
movie_reader.py: error: the following arguments are required: str
$ ./movie_reader.py -h
usage: movie_reader.py [-h] str

Movie Reader

positional arguments:
  str         Input text or file

optional arguments:
  -h, --help  show this help message and exit
$ ./movie_reader.py 'Foo, bar!'
Foo, bar!
$ ./movie_reader.py ../inputs/fox.txt
The quick brown fox jumps over the lazy dog.
\end{verbatim}

\pagebreak

\hypertarget{solution-6}{%
\section{Solution}\label{solution-6}}

\begin{verbatim}
#!/usr/bin/env python3
"""
Author : Ken Youens-Clark <kyclark@gmail.com>
Date   : 2019-05-29
Purpose: Movie Reader
"""

import argparse
import os
import sys
import time


# --------------------------------------------------
def get_args():
    """Get command-line arguments"""

    parser = argparse.ArgumentParser(
        description='Movie Reader',
        formatter_class=argparse.ArgumentDefaultsHelpFormatter)

    parser.add_argument('text',
                        metavar='str',
                        help='Input text or file')

    return parser.parse_args()


# --------------------------------------------------
def main():
    """Make a jazz noise here"""

    args = get_args()
    text = args.text

    if os.path.isfile(text):
        text = open(text).read()

    for line in text.splitlines():
        for char in line:
            print(char, end='')
            time.sleep(.5 if char in '.!?\n' else .2 if char in ',:;' else .05)
            sys.stdout.flush()

        print()

# --------------------------------------------------
if __name__ == '__main__':
    main()
\end{verbatim}

\pagebreak

\hypertarget{chapter-8}{%
\section{Chapter 8}\label{chapter-8}}

\hypertarget{palindromes}{%
\section{Palindromes}\label{palindromes}}

Write a Python program called \texttt{palindromic.py} that will find
words that are palindromes in positional argument which is either a
string or a file name.

\pagebreak

\hypertarget{solution-7}{%
\section{Solution}\label{solution-7}}

\begin{verbatim}
#!/usr/bin/env python3
"""
Author : Ken Youens-Clark <kyclark@gmail.com>
Date   : 2019-05-29
Purpose: Find palindromes in text
"""

import argparse
import os
import re
import sys


# --------------------------------------------------
def get_args():
    """Get command-line arguments"""

    parser = argparse.ArgumentParser(
        description='Find palindromes in text',
        formatter_class=argparse.ArgumentDefaultsHelpFormatter)

    parser.add_argument('text', metavar='str', help='Input text or file')

    parser.add_argument('-m',
                        '--min',
                        metavar='int',
                        type=int,
                        help='Minimum word length',
                        default=3)

    return parser.parse_args()


# --------------------------------------------------
def main():
    """Make a jazz noise here"""

    args = get_args()
    text = args.text
    min_length = args.min

    if os.path.isfile(text):
        text = open(text).read()

    for line in text.splitlines():
        for word in re.split('(\W+)', line.lower()):
            if len(word) >= min_length:
                rev = ''.join(reversed(word))
                if rev == word:
                    print(word)


# --------------------------------------------------
if __name__ == '__main__':
    main()
\end{verbatim}

\pagebreak

\hypertarget{chapter-9}{%
\section{Chapter 9}\label{chapter-9}}

\hypertarget{ransom}{%
\section{Ransom}\label{ransom}}

Create a Python program called \texttt{ransom.py} that will randomly
capitalize the letters in a given word or phrase. The input text may
also name a file in which case the text should come from the file. The
program should take a \texttt{-s\textbar{}-\/-seed} argument for the
\texttt{random.seed} to control randomness for the test suite. It should
also respond to \texttt{-h\textbar{}-\/-help} for usage.

\hypertarget{expected-behavior-2}{%
\section{Expected Behavior}\label{expected-behavior-2}}

\begin{verbatim}
$ ./ransom.py
usage: ransom.py [-h] [-s int] str
ransom.py: error: the following arguments are required: str
$ ./ransom.py -h
usage: ransom.py [-h] [-s int] str

Ransom Note

positional arguments:
  str                 Input text or file

optional arguments:
  -h, --help          show this help message and exit
  -s int, --seed int  Random seed (default: None)
$ cat fox.txt
The quick brown fox jumps over the lazy dog.
$ ./ransom.py fox.txt
the quiCK bROWn fOx JUMps OveR tHe LAzy Dog.
$ ./ransom.py -s 2 'The quick brown fox jumps over the lazy dog.'
the qUIck BROWN fOX JUmps ovEr ThE LAZY DOg.
\end{verbatim}

\pagebreak

\hypertarget{solution-8}{%
\section{Solution}\label{solution-8}}

\begin{verbatim}
#!/usr/bin/env python3
"""
Author : Ken Youens-Clark <kyclark@gmail.com>
Date   : 2019-05-02
Purpose: Ransom Note
"""

import argparse
import os
import random
import sys


# --------------------------------------------------
def get_args():
    """get command-line arguments"""
    parser = argparse.ArgumentParser(
        description='Ransom Note',
        formatter_class=argparse.ArgumentDefaultsHelpFormatter)

    parser.add_argument('text', metavar='str', help='Input text or file')

    parser.add_argument(
        '-s',
        '--seed',
        help='Random seed',
        metavar='int',
        type=int,
        default=None)

    return parser.parse_args()


# --------------------------------------------------
def warn(msg):
    """Print a message to STDERR"""
    print(msg, file=sys.stderr)


# --------------------------------------------------
def die(msg='Something bad happened'):
    """warn() and exit with error"""
    warn(msg)
    sys.exit(1)


# --------------------------------------------------
def main():
    """Make a jazz noise here"""
    args = get_args()

    random.seed(args.seed)

    text = args.text
    if os.path.isfile(text):
        text = open(text).read()

    #ransom = []
    #for char in text:
    #    ransom.append(char.upper() if random.choice([0, 1]) else char.lower())

    #ransom = [c.upper() if random.choice([0, 1]) else c.lower() for c in text]

    #ransom = map(lambda c: c.upper() if random.choice([0, 1]) else c.lower(),
    #             text)

    f = lambda c: c.upper() if random.choice([0, 1]) else c.lower()
    ransom = map(f, text)

    print(''.join(ransom))


# --------------------------------------------------
if __name__ == '__main__':
    main()
\end{verbatim}

\pagebreak

\hypertarget{chapter-10}{%
\section{Chapter 10}\label{chapter-10}}

\hypertarget{simple-rhymer}{%
\section{Simple Rhymer}\label{simple-rhymer}}

Write a Python program called \texttt{rhymer.py} that will create new
words by removing the consonant(s) from the beginning of the word and
then creating new words by prefixing the remainder with all the
consonants and clusters that were not at the beginning. That is, prefix
with all the consonants in the alphabet plus these clusters:

\begin{verbatim}
bl br ch cl cr dr fl fr gl gr pl pr sc sh sk sl sm sn sp 
st sw th tr tw wh wr sch scr shr sph spl spr squ str thr
\end{verbatim}

\begin{verbatim}
$ ./rhymer.py
usage: rhymer.py [-h] str
rhymer.py: error: the following arguments are required: str
$ ./rhymer.py -h
usage: rhymer.py [-h] str

Make rhyming "words"

positional arguments:
  str         A word

optional arguments:
  -h, --help  show this help message and exit
$ ./rhymer.py apple
Word "apple" must start with consonants  
$ ./rhymer.py take | head
bake
cake
dake
fake
gake
hake
jake
kake
lake
make
\end{verbatim}

\pagebreak

\hypertarget{solution-9}{%
\section{Solution}\label{solution-9}}

\begin{verbatim}
#!/usr/bin/env python3
"""
Author : Ken Youens-Clark <kyclark@gmail.com>
Date   : 2019-05-22
Purpose: Make rhyming "words"
"""

import argparse
import re
import string
import sys
from dire import die


# --------------------------------------------------
def get_args():
    """get command-line arguments"""
    parser = argparse.ArgumentParser(
        description='Make rhyming "words"',
        formatter_class=argparse.ArgumentDefaultsHelpFormatter)

    parser.add_argument('word', metavar='str', help='A word')

    return parser.parse_args()


# --------------------------------------------------
def main():
    """Make a jazz noise here"""
    args = get_args()
    word = args.word

    vowels = 'aeiou'
    if word[0] in vowels:
        die('Word "{}" must start with consonants'.format(word))

    consonants = [c for c in string.ascii_lowercase if c not in 'aeiou']
    match = re.match('^([' + ''.join(consonants) + ']+)(.+)', word)

    clusters = ('bl br ch cl cr dr fl fr gl gr pl pr sc '
                'sh sk sl sm sn sp st sw th tr tw wh wr '
                'sch scr shr sph spl spr squ str thr').split()

    if match:
        start, rest = match.group(1), match.group(2)
        for c in filter(lambda c: c != start, consonants + clusters):
            print(c + rest)


# --------------------------------------------------
if __name__ == '__main__':
    main()
\end{verbatim}

\pagebreak

\hypertarget{chapter-11}{%
\section{Chapter 11}\label{chapter-11}}

\hypertarget{rock-paper-scissors}{%
\section{Rock, Paper, Scissors}\label{rock-paper-scissors}}

Write a Python program called \texttt{rps.py} that will play the
ever-popular ``Rock, Paper, Scissors'' game.

\begin{verbatim}
$ ./rps.py
1-2-3-Go! [rps|q] r
You: Rock
Me : Scissors
You win. You are a clammy drate-poke.
1-2-3-Go! [rps|q] t
You dysfunctional dew-beater! Please choose from: p, r, s.
1-2-3-Go! [rps|q] p
You: Paper
Me : Rock
You win. You are a dismal gillie-wet-foot.
1-2-3-Go! [rps|q] q
Bye, you imbecilic fopdoodle!
\end{verbatim}

\pagebreak

\hypertarget{solution-10}{%
\section{Solution}\label{solution-10}}

\begin{verbatim}
#!/usr/bin/env python3
"""
Author : Ken Youens-Clark <kyclark@gmail.com>
Date   : 2019-04-09
Purpose: Play rock, paper, scissors
"""

import sys
import random


# --------------------------------------------------
def insult():
    adjective = """
    truculent fatuous vainglorious fatuous petulant moribund jejune
    feckless antiquated rambunctious mundane misshapen glib dreary
    dopey devoid deleterious degrading clammy brazen indiscreet
    indecorous imbecilic dysfunctional dubious drunken disreputable
    dismal dim deficient deceitful damned daft contrary churlish
    catty banal asinine infantile lurid morbid repugnant unkempt
    vapid decrepit malevolent impertinent decrepit grotesque puerile
    """.split()

    noun = """
    abydocomist bedswerver bespawler bobolyne cumberworld dalcop
    dew-beater dorbel drate-poke driggle-draggle fopdoodle fustylugs
    fustilarian gillie-wet-foot gnashgab gobermouch
    gowpenful-o’-anything klazomaniac leasing-monger loiter-sack
    lubberwort muck-spout mumblecrust quisby raggabrash rakefire
    roiderbanks saddle-goose scobberlotcher skelpie-limmer
    smell-feast smellfungus snoutband sorner stampcrab stymphalist
    tallowcatch triptaker wandought whiffle-whaffle yaldson zoilist
    """.split()

    return ' '.join([random.choice(adjective), random.choice(noun)])


# --------------------------------------------------
def main():
    """Play Rock Paper Scissors"""
    valid = set('rps')

    beats = {'r': 's', 's': 'p', 'p': 'r'}
    display = {'r': 'Rock', 'p': 'Paper', 's': 'Scissors'}

    while True:
        play = input('1-2-3-Go! [rps|q] ').lower()

        if play.startswith('q'):
            print('Bye, you {}!'.format(insult()))
            sys.exit(0)

        if play not in valid:
            print('You {}! Please choose from: {}.'.format(
                insult(), ', '.join(sorted(valid))))
            continue

        computer = random.choice(list(valid))

        print('You: {}\nMe : {}'.format(display[play], display[computer]))

        if beats[play] == computer:
            print('You win. You are a {}.'.format(insult()))
        elif beats[computer] == play:
            print('You lose, {}!'.format(insult()))
        else:
            print('Draw, you {}.'.format(insult()))


# --------------------------------------------------
main()
\end{verbatim}

\pagebreak

\hypertarget{chapter-12}{%
\section{Chapter 12}\label{chapter-12}}

\hypertarget{abuse}{%
\section{Abuse}\label{abuse}}

Write a Python program called \texttt{abuse.py} that generates some
\texttt{-n\textbar{}-\/-number} of insults (default 3) by randomly
combining some number of \texttt{-a\textbar{}-\/-adjectives} (default 2)
with a noun (see below). Be sure your program accepts a
\texttt{-s\textbar{}-\/-seed} argument to pass to \texttt{random.seed}.

Adjectives:

bankrupt base caterwauling corrupt cullionly detestable dishonest false
filthsome filthy foolish foul gross heedless indistinguishable infected
insatiate irksome lascivious lecherous loathsome lubbery old peevish
rascaly rotten ruinous scurilous scurvy slanderous sodden-witted
thin-faced toad-spotted unmannered vile wall-eyed

Nouns:

nouns = """ Judas Satan ape ass barbermonger beggar block boy braggart
butt carbuncle coward coxcomb cur dandy degenerate fiend fishmonger fool
gull harpy jack jolthead knave liar lunatic maw milksop minion
ratcatcher recreant rogue scold slave swine traitor varlet villain worm

\begin{verbatim}
$ ./abuse.py -h
usage: abuse.py [-h] [-a int] [-n int] [-s int]

Argparse Python script

optional arguments:
  -h, --help            show this help message and exit
  -a int, --adjectives int
                        Number of adjectives (default: 2)
  -n int, --number int  Number of insults (default: 3)
  -s int, --seed int    Random seed (default: None)
$ ./abuse.py
You slanderous, rotten block!
You lubbery, scurilous ratcatcher!
You rotten, foul liar!
$ ./abuse.py -s 1 -n 2 -a 1
You rotten rogue!
You lascivious ape!
$ ./abuse.py -s 2 -n 4 -a 4
You scurilous, foolish, vile, foul milksop!
You cullionly, lubbery, heedless, filthy lunatic!
You foul, lecherous, infected, slanderous degenerate!
You base, ruinous, slanderous, false liar!
\end{verbatim}

\hypertarget{skills-1}{%
\section{Skills}\label{skills-1}}

\begin{itemize}
\tightlist
\item
  Setting random seed from argument
\item
  Random selecting/sampling from a list
\item
  Iterating through a loop a defined number of times
\item
  Formatting string output
\end{itemize}

\pagebreak

\hypertarget{solution-11}{%
\section{Solution}\label{solution-11}}

\begin{verbatim}
#!/usr/bin/env python3
"""
Author:  Ken Youens-Clark <kyclark@gmail.com>
Date   : 2019-05-17
Purpose: Shakespearean insult generator
"""

import argparse
import random
import sys

adjectives = """
bankrupt base caterwauling corrupt cullionly detestable dishonest
false filthsome filthy foolish foul gross heedless indistinguishable
infected insatiate irksome lascivious lecherous loathsome lubbery old
peevish rascaly rotten ruinous scurilous scurvy slanderous
sodden-witted thin-faced toad-spotted unmannered vile wall-eyed
""".strip().split()

nouns = """
Judas Satan ape ass barbermonger beggar block boy braggart butt
carbuncle coward coxcomb cur dandy degenerate fiend fishmonger fool
gull harpy jack jolthead knave liar lunatic maw milksop minion
ratcatcher recreant rogue scold slave swine traitor varlet villain worm
""".strip().split()


# --------------------------------------------------
def get_args():
    """get command-line arguments"""
    parser = argparse.ArgumentParser(
        description='Argparse Python script',
        formatter_class=argparse.ArgumentDefaultsHelpFormatter)

    parser.add_argument('-a',
                        '--adjectives',
                        help='Number of adjectives',
                        metavar='int',
                        type=int,
                        default=2)

    parser.add_argument('-n',
                        '--number',
                        help='Number of insults',
                        metavar='int',
                        type=int,
                        default=3)

    parser.add_argument('-s',
                        '--seed',
                        help='Random seed',
                        metavar='int',
                        type=int,
                        default=None)

    return parser.parse_args()


# --------------------------------------------------
def warn(msg):
    """Print a message to STDERR"""
    print(msg, file=sys.stderr)


# --------------------------------------------------
def die(msg='Something bad happened'):
    """warn() and exit with error"""
    warn(msg)
    sys.exit(1)


# --------------------------------------------------
def main():
    """Make a jazz noise here"""
    args = get_args()
    num_adj = args.adjectives
    num_insults = args.number

    random.seed(args.seed)

    for _ in range(num_insults):
        adjs = random.sample(adjectives, k=num_adj)
        noun = random.choice(nouns)
        print('You {} {}!'.format(', '.join(adjs), noun))


# --------------------------------------------------
if __name__ == '__main__':
    main()
\end{verbatim}

\pagebreak

\hypertarget{chapter-13}{%
\section{Chapter 13}\label{chapter-13}}

\hypertarget{bacronym}{%
\section{BACRONYM}\label{bacronym}}

Write a Python program called \texttt{bacronym.py} that takes an string
like ``FBI'' and retrofits some \texttt{-n\textbar{}-\/-number} (default
5) of acronyms by reading a \texttt{-w\textbar{}-\/-wordlist} argument
(defualt ``/usr/share/dict/words''), skipping over words to
\texttt{-e\textbar{}-\/-exclude} (default ``a, an, the'') and randomly
selecting words that start with each of the letters. Be sure to include
a \texttt{-s\textbar{}-\/-seed} argument (default \texttt{None}) to pass
to \texttt{random.seed} for the test suite.

\begin{verbatim}
$ ./bacronym.py
usage: bacronym.py [-h] [-n NUM] [-w STR] [-x STR] [-s INT] STR
bacronym.py: error: the following arguments are required: STR
$ ./bacronym.py -h
usage: bacronym.py [-h] [-n NUM] [-w STR] [-x STR] [-s INT] STR

Explain acronyms

positional arguments:
  STR                   Acronym

optional arguments:
  -h, --help            show this help message and exit
  -n NUM, --num NUM     Maximum number of definitions (default: 5)
  -w STR, --wordlist STR
                        Dictionary/word file (default: /usr/share/dict/words)
  -x STR, --exclude STR
                        List of words to exclude (default: a,an,the)
  -s INT, --seed INT    Random seed (default: None)
$ ./bacronym.py FBI -s 1
FBI =
 - Fecundity Brokage Imitant
 - Figureless Basketmaking Ismailite
 - Frumpery Bonedog Irregardless
 - Foxily Blastomyces Inedited
 - Fastland Bouncingly Idiospasm
\end{verbatim}

\hypertarget{skills-2}{%
\section{Skills}\label{skills-2}}

\begin{itemize}
\tightlist
\item
  Using \texttt{argparse}
\item
  Reading words from a file into a list
\item
  Randomly selecting words from a list
\item
  Formatting string output
\end{itemize}

\pagebreak

\hypertarget{solution-12}{%
\section{Solution}\label{solution-12}}

\begin{verbatim}
#!/usr/bin/env python3
"""
Purpose: Make guesses about acronyms
Author:  Ken Youens-Clark <kyclark@gmail.com>
Date:    2019-05-19
"""

import argparse
import sys
import os
import random
import re
from collections import defaultdict


# --------------------------------------------------
def get_args():
    """get arguments"""
    parser = argparse.ArgumentParser(
        description='Explain acronyms',
        formatter_class=argparse.ArgumentDefaultsHelpFormatter)

    parser.add_argument('acronym', help='Acronym', type=str, metavar='STR')

    parser.add_argument('-n',
                        '--num',
                        help='Maximum number of definitions',
                        type=int,
                        metavar='NUM',
                        default=5)

    parser.add_argument('-w',
                        '--wordlist',
                        help='Dictionary/word file',
                        type=str,
                        metavar='STR',
                        default='/usr/share/dict/words')

    parser.add_argument('-x',
                        '--exclude',
                        help='List of words to exclude',
                        type=str,
                        metavar='STR',
                        default='a,an,the')

    parser.add_argument('-s',
                        '--seed',
                        help='Random seed',
                        type=int,
                        metavar='INT',
                        default=None)

    return parser.parse_args()


# --------------------------------------------------
def main():
    """main"""
    args = get_args()
    acronym = args.acronym
    wordlist = args.wordlist
    limit = args.num
    goodword = r'^[a-z]{2,}$'
    badwords = set(re.split(r'\s*,\s*', args.exclude.lower()))

    random.seed(args.seed)

    if not re.match(goodword, acronym.lower()):
        print('"{}" must be >1 in length, only use letters'.format(acronym))
        sys.exit(1)

    if not os.path.isfile(wordlist):
        print('"{}" is not a file.'.format(wordlist))
        sys.exit(1)

    seen = set()
    words_by_letter = defaultdict(list)
    for word in open(wordlist).read().lower().split():
        clean = re.sub('[^a-z]', '', word)
        if not clean:  # nothing left?
            continue

        if re.match(goodword,
                    clean) and clean not in seen and clean not in badwords:
            seen.add(clean)
            words_by_letter[clean[0]].append(clean)

    len_acronym = len(acronym)
    definitions = []
    for i in range(0, limit):
        definition = []
        for letter in acronym.lower():
            possible = words_by_letter.get(letter, [])
            if len(possible) > 0:
                definition.append(
                    random.choice(possible).title() if possible else '?')

        if len(definition) == len_acronym:
            definitions.append(' '.join(definition))

    if len(definitions) > 0:
        print(acronym.upper() + ' =')
        for definition in definitions:
            print(' - ' + definition)
    else:
        print('Sorry I could not find any good definitions')


# --------------------------------------------------
if __name__ == '__main__':
    main()
\end{verbatim}

\pagebreak

\hypertarget{chapter-14}{%
\section{Chapter 14}\label{chapter-14}}

\hypertarget{python-blackjack-game}{%
\section{Python Blackjack Game}\label{python-blackjack-game}}

Write a Python program called \texttt{blackjack.py} that plays an
abbreviated game of Blackjack. You will need to \texttt{import\ random}
to get random cards from a deck you will construct, and so your program
will need to accept a \texttt{-s\textbar{}-\/-seed} that will set
\texttt{random.seed()} with the value that is passed in so that the test
suite will work. The other arguments you will accept are two flags
(Boolean values) of \texttt{-p\textbar{}-\/-player\_hits} and
\texttt{-d\textbar{}-\/-dealer\_hits}. As usual, you will also have a
\texttt{-h\textbar{}-\/-help} option for usage statement.

To play the game, the user will run the program and will see a display
of what cards the dealer has (noted ``D'') and what cards the player has
(noted ``P'') along with a sum of the values of the cards. In Blackjack,
number cards are worth their value, face cards are worth 10, and the Ace
will be worth 1 for our game (though in the real game it can alternate
between 1 and 11).

To create your deck of cards, you will need to use the Unicode symbols
for the suites ( ♥ ♠ ♣ ♦ ) {[}which won't display in the PDF, so consult
the Markdown file{]}.

Combine these with the numbers 2-10 and the letters ``A'', ``J'', ``Q,''
and ``K'' (hint: look at \texttt{itertools.product}). Because your game
will use randomness, you will need to sort your deck and then use the
\texttt{random.shuffle} method so that your cards will be in the correct
order to pass the tests.

When you make the initial deal, keep in mind how cards are actually
dealt -- first one card to each of the players, then one to the dealer,
then the players, then the dealer, etc. You might be tempted to use
\texttt{random.choice} or something like that to select your cards, but
you need to keep in mind that you are modeling an actual deck and so
selected cards should no longer be present in the deck. If the
\texttt{-p\textbar{}-\/-player\_htis} flag is present, deal an
additional card to the player; likewise with the
\texttt{-d\textbar{}-\/-dealer\_hits} flag.

After displaying the hands, the code should:

\begin{enumerate}
\def\labelenumi{\arabic{enumi}.}
\tightlist
\item
  Check if the player has more than 21; if so, print `Player busts! You
  lose, loser!' and exit(0)
\item
  Check if the dealer has more than 21; if so, print `Dealer busts.' and
  exit(0)
\item
  Check if the player has exactly 21; if so, print `Player wins. You
  probably cheated.' and exit(0)
\item
  Check if the dealer has exactly 21; if so, print `Dealer wins!' and
  exit(0)
\item
  If the either the dealer or the player has less than 18, you should
  indicate ``X should hit.''
\end{enumerate}

NB: Look at the Markdown format to see the actual output as the suites
won't display in the PDF version!

\begin{verbatim}
$ ./blackjack.py
D [11]: ♠J ♥A
P [18]: ♦8 ♠10
Dealer should hit.
$ ./blackjack.py
D [13]: ♥3 ♦J
P [16]: ♥6 ♦10
Dealer should hit.
Player should hit.
$ ./blackjack.py -s 5
D [ 5]: ♣4 ♣A
P [19]: ♦10 ♦9
Dealer should hit.
$ ./blackjack.py -s 3 -p
D [19]: ♥K ♠9
P [22]: ♣3 ♥9 ♣J
Player busts! You lose, loser!
$ ./blackjack.py -s 15 -p
D [19]: ♠10 ♦9
P [21]: ♣10 ♥8 ♠3
Player wins. You probably cheated.
\end{verbatim}

\pagebreak

\hypertarget{solution-13}{%
\section{Solution}\label{solution-13}}

\begin{verbatim}
#!/usr/bin/env python3
"""
Author : Ken Youens-Clark <kyclark@gmail.com>
Date   : 2019-03-15
Purpose: Rock the Casbah
"""

import argparse
import random
import re
import sys
from itertools import product


# --------------------------------------------------
def get_args():
    """get command-line arguments"""
    parser = argparse.ArgumentParser(
        description='Argparse Python script',
        formatter_class=argparse.ArgumentDefaultsHelpFormatter)

    parser.add_argument(
        '-s',
        '--seed',
        help='Random seed',
        metavar='int',
        type=int,
        default=None)

    parser.add_argument(
        '-d',
        '--dealer_hits',
        help='Dealer hits',
        action='store_true')

    parser.add_argument(
        '-p',
        '--player_hits',
        help='Player hits',
        action='store_true')

    return parser.parse_args()


# --------------------------------------------------
def warn(msg):
    """Print a message to STDERR"""
    print(msg, file=sys.stderr)


# --------------------------------------------------
def die(msg='Something bad happened'):
    """warn() and exit with error"""
    warn(msg)
    sys.exit(1)

# --------------------------------------------------
def bail(msg):
    """print() and exit(0)"""
    print(msg)
    sys.exit(0)

# --------------------------------------------------
def card_value(card):
    """card to numeric value"""
    val = card[1:]
    faces = {'A': 1, 'J': 10, 'Q': 10, 'K': 10}
    if val.isdigit():
        return int(val)
    elif val in faces:
        return faces[val]
    else:
        die('Unknown card value for "{}"'.format(card))


# --------------------------------------------------
def main():
    """Make a jazz noise here"""
    args = get_args()

    random.seed(args.seed)

    # seed = args.seed
    # if seed is not None:
    #     random.seed(seed)

    suites = list('♥♠♣♦')
    values = list(range(2, 11)) + list('AJQK')
    cards = sorted(map(lambda t: '{}{}'.format(*t), product(suites, values)))
    random.shuffle(cards)

    p1, d1, p2, d2 = cards.pop(), cards.pop(), cards.pop(), cards.pop()
    player = [p1, p2]
    dealer = [d1, d2]

    if args.player_hits:
        player.append(cards.pop())

    if args.dealer_hits:
        dealer.append(cards.pop())

    player_hand = sum(map(card_value, player))
    dealer_hand = sum(map(card_value, dealer))

    print('D [{:2}]: {}'.format(dealer_hand, ' '.join(dealer)))
    print('P [{:2}]: {}'.format(player_hand, ' '.join(player)))

    if player_hand > 21:
        bail('Player busts! You lose, loser!')
    elif dealer_hand > 21:
        bail('Dealer busts.')
    elif player_hand == 21:
        bail('Player wins. You probably cheated.')
    elif dealer_hand == 21:
        bail('Dealer wins!')

    if dealer_hand < 18:
        print('Dealer should hit.')

    if player_hand < 18:
        print('Player should hit.')


# --------------------------------------------------
if __name__ == '__main__':
    main()
\end{verbatim}

\pagebreak

\hypertarget{chapter-15}{%
\section{Chapter 15}\label{chapter-15}}

\hypertarget{family-tree}{%
\section{Family Tree}\label{family-tree}}

Write a program called \texttt{tree.py} that will take an input file as
a single positional argument and produce a graph of the family tree
described therein. The file can have only three kinds of statements:

\begin{enumerate}
\def\labelenumi{\arabic{enumi}.}
\tightlist
\item
  \texttt{INITIALS\ =\ Full\ Name}
\item
  \texttt{person1\ married\ person2}
\item
  \texttt{person1\ and\ person2\ begat\ child1{[},\ child2...{]}}
\end{enumerate}

Use the \texttt{graphviz} module to generate a graph like the
\texttt{kyc.gv.pdf} included here that was generated from the following
input:

\begin{verbatim}
$ cat tudor.txt
H7 = Henry VII
EOY = Elizabeth of York
H8 = Henry VIII
COA = Catherine of Aragon
AB = Anne Boleyn
JS = Jane Seymour
AOC = Anne of Cleves
CH = Catherine Howard
CP = Catherine Parr
HDC = Henry, Duke of Cornwall
M1 = Mary I
E1 = Elizabeth I
E6 = Edward VI

H7 married EOY
H7 and EOY begat H8
H8 married COA
H8 married AB
H8 married JS
H8 married AOC
H8 married CH
H8 married CP
H8 and COA begat HDC, M1
H8 and AB begat E1
H8 and JS begat E6
$ ./tree.py tudor.txt
Done, see output in "tudor.txt.gv".
\end{verbatim}

\pagebreak

\hypertarget{chapter-17}{%
\section{Chapter 17}\label{chapter-17}}

\hypertarget{guessing-game}{%
\section{Guessing Game}\label{guessing-game}}

Write a Python program called \texttt{guess.py} that plays a guessing
game for a number between a \texttt{-m\textbar{}-\/-min} and
\texttt{-x\textbar{}-\/-max} value (default 1 and 50, respectively) with
a limited number of \texttt{-g\textbar{}-\/-guesses} (default 5).
Complain if either \texttt{-\/-min} or \texttt{-\/-guesses} is less than
1. Accept a \texttt{-s\textbar{}-\/-seed} for \texttt{random.seed}. If
the user guesses something that is not a number, complain about it.

The game is intended to actually be interactive, which makes it
difficult to test. Here is how it should look in interactive mode:

\begin{verbatim}
$ ./guess.py -s 1
Guess a number between 1 and 50 (q to quit): 25
"25" is too high.
Guess a number between 1 and 50 (q to quit): foo
"foo" is not a number.
Guess a number between 1 and 50 (q to quit): 12
"12" is too high.
Guess a number between 1 and 50 (q to quit): 6
"6" is too low.
Guess a number between 1 and 50 (q to quit): 9
"9" is correct. You win!
\end{verbatim}

Because I want to be able to write a test for this, I also want the
program to accept an \texttt{-i\textbar{}-\/-inputs} option so that the
game can also be played exactly the same but without the prompts for
input:

\begin{verbatim}
$ ./guess.py -s 1 -i 25 foo 12 6 9
"25" is too high.
"foo" is not a number.
"12" is too high.
"6" is too low.
"9" is correct. You win!
\end{verbatim}

You should be able to handle this in your inifinite game loop.

\pagebreak

\hypertarget{solution-14}{%
\section{Solution}\label{solution-14}}

\begin{verbatim}
#!/usr/bin/env python3
"""
Author:  Ken Youens-Clark <kyclark@gmail.com>
Purpose: Guess-the-number game
"""

import argparse
import random
import re
import sys
from dire import die


# --------------------------------------------------
def get_args():
    """get args"""
    parser = argparse.ArgumentParser(
        description='Guessing game',
        formatter_class=argparse.ArgumentDefaultsHelpFormatter)

    parser.add_argument('-m',
                        '--min',
                        help='Minimum value',
                        metavar='int',
                        type=int,
                        default=1)

    parser.add_argument('-x',
                        '--max',
                        help='Maximum value',
                        metavar='int',
                        type=int,
                        default=50)

    parser.add_argument('-g',
                        '--guesses',
                        help='Number of guesses',
                        metavar='int',
                        type=int,
                        default=5)

    parser.add_argument('-s',
                        '--seed',
                        help='Random seed',
                        metavar='int',
                        type=int,
                        default=None)

    parser.add_argument('-i',
                        '--inputs',
                        help='Inputs',
                        metavar='str',
                        type=str,
                        nargs='+',
                        default=[])

    return parser.parse_args()


# --------------------------------------------------
def main():
    """main"""
    args = get_args()
    low = args.min
    high = args.max
    guesses_allowed = args.guesses
    inputs = args.inputs
    random.seed(args.seed)

    if low < 1:
        die('--min "{}" cannot be lower than 1'.format(low))

    if guesses_allowed < 1:
        die('--guesses "{}" cannot be lower than 1'.format(guesses_allowed))

    if low > high:
        die('--min "{}" is higher than --max "{}"'.format(low, high))

    secret = random.randint(low, high)
    prompt = 'Guess a number between {} and {} (q to quit): '.format(low, high)
    num_guesses = 0

    while True:
        guess = inputs.pop(0) if inputs else input(prompt)
        num_guesses += 1

        if re.match('q(uit)?', guess.lower()):
            print('Now you will never know the answer.')
            sys.exit()

        # Method 1: test if the guess is a digit
        if not guess.isdigit():
            print('"{}" is not a number.'.format(guess))
            continue
        num = int(guess)

        # Method 2: try/except
        num = 0
        try:
            num = int(guess)
        except:
            warn('"{}" is not an integer'.format(guess))
            continue

        if not low <= num <= high:
            print('Number "{}" is not in the allowed range'.format(num))
        elif num == secret:
            print('"{}" is correct. You win!'.format(num))
            break
        else:
            print('"{}" is too {}.'.format(num,
                                           'low' if num < secret else 'high'))

        if num_guesses >= guesses_allowed:
            print(
                'Too many guesses, loser! The number was "{}."'.format(secret))
            sys.exit(1)


# --------------------------------------------------
if __name__ == '__main__':
    main()
\end{verbatim}

\pagebreak

\hypertarget{chapter-18}{%
\section{Chapter 18}\label{chapter-18}}

\hypertarget{kentucky-fryer}{%
\section{Kentucky Fryer}\label{kentucky-fryer}}

Write a Python program called \texttt{fryer.py} that reads some input
text from a single positional argument on the command line (which could
be a file to read) and transforms the text by dropping the ``g'' from
words two-syllable words ending in ``-ing'' and also changes ``you'' to
``y'all''. Be mindful to keep the case the same on the first letter,
e.g, ``You'' should become ``Y'all,'' ``Hunting'' should become
``Huntin'''.

\begin{verbatim}
$ ./fryer.py
usage: fryer.py [-h] str
fryer.py: error: the following arguments are required: str
$ ./fryer.py -h
usage: fryer.py [-h] str

Southern fry text

positional arguments:
  str         Input text or file

optional arguments:
  -h, --help  show this help message and exit
$ ./fryer.py you
y'all
$ ./fryer.py Fishing
Fishin'
$ ./fryer.py string
string
$ cat tests/input1.txt
So I was fixing to ask him, "Do you want to go fishing?" I was dying
to go for a swing and maybe do some swimming, too.
$ ./fryer.py tests/input1.txt
So I was fixin' to ask him, "Do y'all want to go fishing?" I was dyin'
to go for a swing and maybe do some swimmin', too.
\end{verbatim}

\pagebreak

\hypertarget{solution-15}{%
\section{Solution}\label{solution-15}}

\begin{verbatim}
#!/usr/bin/env python3
"""
Author : Ken Youens-Clark <kyclark@gmail.com>
Date   : 2019-05-21
Purpose: Southern fry text
"""

import argparse
import os
import re
import sys


# --------------------------------------------------
def get_args():
    """get command-line arguments"""
    parser = argparse.ArgumentParser(
        description='Southern fry text',
        formatter_class=argparse.ArgumentDefaultsHelpFormatter)

    parser.add_argument('text', metavar='str', help='Input text or file')

    return parser.parse_args()


# --------------------------------------------------
def fry(word):
    """
    Drop the 'g' from '-ing' words, change "you" to "y'all"
    """

    ing_word = re.search('(.+)ing([:;,.?])?$', word)
    you = re.match('([Yy])ou$', word)

    if ing_word:
        prefix = ing_word.group(1)
        if re.search('[aeiouy]', prefix):
            return prefix + "in'" + (ing_word.group(2) or '')
    elif you:
        return you.group(1) + "'all"

    return word


# --------------------------------------------------
def main():
    """Make a jazz noise here"""

    args = get_args()
    text = args.text

    if os.path.isfile(text):
        text = open(text).read()

    for line in text.splitlines():
        print(' '.join(map(fry, line.rstrip().split())))


# --------------------------------------------------
if __name__ == '__main__':
    main()
\end{verbatim}

\pagebreak

\hypertarget{chapter-20}{%
\section{Chapter 20}\label{chapter-20}}

\hypertarget{markov-chains-for-words}{%
\section{Markov Chains for Words}\label{markov-chains-for-words}}

Write a Python program called \texttt{markov.py} that uses the Markov
chain algorithm to generate new words from a set of training files. The
program should take one or more positional arguments which are files
that you read, word-by-word, and note the options of letters after a
given \texttt{-k\textbar{}-\/-kmer\_size} (default \texttt{2}) grouping
of letters. E.g., in the word ``alabama'' with \texttt{k=1}, the
frequency table will look like:

\begin{verbatim}
a = l, b, m
l = a
b = a
m = a
\end{verbatim}

That is, given this training set, if you started with \texttt{l} you
could only choose an \texttt{a}, but if you have \texttt{a} then you
could choose \texttt{l}, \texttt{b}, or \texttt{m}.

The program should generate \texttt{-n\textbar{}-\/-num\_words} words
(default \texttt{10}), each a random size between \texttt{k} + 2 and a
\texttt{-m\textbar{}-\/-max\_word} size (default \texttt{12}). Be sure
to accept \texttt{-s\textbar{}-\/-seed} to pass to \texttt{random.seed}.
My solution also takes a \texttt{-d\textbar{}-\/-debug} flag that will
emit debug messages to \texttt{.log} for you to inspect.

Chose the best words and create definitions for them:

\begin{itemize}
\tightlist
\item
  yulcogicism: the study of Christmas gnostics
\item
  umjamp: skateboarding trick
\item
  callots: insignia of officers in Greek army
\item
  urchenev: fungal growth found under cobblestones
\end{itemize}

\begin{verbatim}
$ ./markov.py
usage: markov.py [-h] [-n int] [-k int] [-m int] [-s int] [-d] FILE [FILE ...]
markov.py: error: the following arguments are required: FILE
$ ./markov.py -h
usage: markov.py [-h] [-n int] [-k int] [-m int] [-s int] [-d] FILE [FILE ...]

Markov chain for characters/words

positional arguments:
  FILE                  Training file(s)

optional arguments:
  -h, --help            show this help message and exit
  -n int, --num_words int
                        Number of words to generate (default: 10)
  -k int, --kmer_size int
                        Kmer size (default: 2)
  -m int, --max_word int
                        Max word length (default: 12)
  -s int, --seed int    Random seed (default: None)
  -d, --debug           Debug to ".log" (default: False)
$ ./markov.py /usr/share/dict/words -s 1
  1: oveli
  2: uming
  3: uylatiteda
  4: owsh
  5: uuse
  6: ismandl
  7: efortai
  8: eyhopy
  9: auretrab
 10: ozogralach
$ ./markov.py ../inputs/const.txt -s 2 -k 3
  1: romot
  2: leasonsusp
  3: gdoned
  4: bunablished
  5: neithere
  6: achmen
  7: reason
  8: nmentyone
  9: effereof
 10: eipts
\end{verbatim}

\pagebreak

\hypertarget{solution-16}{%
\section{Solution}\label{solution-16}}

\begin{verbatim}
#!/usr/bin/env python3
"""
Author : Ken Youens-Clark <kyclark@gmail.com>
Date   : 2019-05-24
Purpose: Markov chain for characters/words
"""

import argparse
import logging
import os
import random
import re
import sys
from collections import defaultdict


# --------------------------------------------------
def get_args():
    """Get command-line arguments"""

    parser = argparse.ArgumentParser(
        description='Markov chain for characters/words',
        formatter_class=argparse.ArgumentDefaultsHelpFormatter)

    parser.add_argument('file',
                        metavar='FILE',
                        nargs='+',
                        help='Training file(s)')

    parser.add_argument('-n',
                        '--num_words',
                        help='Number of words to generate',
                        metavar='int',
                        type=int,
                        default=10)

    parser.add_argument('-k',
                        '--kmer_size',
                        help='Kmer size',
                        metavar='int',
                        type=int,
                        default=2)

    parser.add_argument('-m',
                        '--max_word',
                        help='Max word length',
                        metavar='int',
                        type=int,
                        default=12)

    parser.add_argument('-s',
                        '--seed',
                        help='Random seed',
                        metavar='int',
                        type=int,
                        default=None)

    parser.add_argument('-d',
                        '--debug',
                        help='Debug to ".log"',
                        action='store_true')

    return parser.parse_args()


# --------------------------------------------------
def main():
    """Make a jazz noise here"""

    args = get_args()
    k = args.kmer_size
    random.seed(args.seed)

    logging.basicConfig(
        filename='.log',
        filemode='w',
        level=logging.DEBUG if args.debug else logging.CRITICAL)

    # debate use of set/list in terms of letter frequencies
    chains = defaultdict(list)
    for file in args.file:
        for line in open(file):
            for word in line.lower().split():
                word = re.sub('[^a-z]', '', word)
                for i in range(0, len(word) - k):
                    kmer = word[i:i + k + 1]
                    chains[kmer[:-1]].append(kmer[-1])

    logging.debug(chains)

    kmers = list(chains.keys())
    starts = set()

    for i in range(1, args.num_words + 1):
        word = ''
        while not word:
            kmer = random.choice(kmers)
            if not kmer in starts and chains[kmer] and re.search(
                    '[aeiou]', kmer):
                starts.add(kmer)
                word = kmer

        length = random.choice(range(k + 2, args.max_word))
        logging.debug('Make a word {} long starting with "{}"'.format(
            length, word))
        while len(word) < length:
            if not chains[kmer]: break
            char = random.choice(list(chains[kmer]))
            logging.debug('char = "{}"'.format(char))
            word += char
            kmer = kmer[1:] + char

        logging.debug('word = "{}"'.format(word))
        print('{:3}: {}'.format(i, word))


# --------------------------------------------------
if __name__ == '__main__':
    main()
\end{verbatim}

\pagebreak

\hypertarget{chapter-21}{%
\section{Chapter 21}\label{chapter-21}}

\hypertarget{pig-latin}{%
\section{Pig Latin}\label{pig-latin}}

Write a Python program named \texttt{piggie.py} that takes one or more
file names as positional arguments and converts all the words in them
into ``Pig Latin'' (see rules below). Write the output to a directory
given with the flags \texttt{-o\textbar{}-\/-outdir} (default
\texttt{out-yay}) using the same basename as the input file, e.g.,
\texttt{input/foo.txt} would be written to \texttt{out-yay/foo.txt}.

if a file argument names a non-existent file, print a warning to STDERR
and skip that file. If the output directory does not exist, create it.

\hypertarget{pig-latin-rules}{%
\section{Pig Latin Rules}\label{pig-latin-rules}}

\begin{enumerate}
\def\labelenumi{\arabic{enumi}.}
\tightlist
\item
  If the word begins with consonants, e.g., ``k'' or ``ch'', move them
  to the end of the word and append ``ay'' so that ``mouse'' becomes
  ``ouse-may'' and ``chair'' becomes ``air-chay.''
\item
  If the word begins with a vowel, simple append ``-yay'' to the end, so
  ``apple'' is ``apple-yay.''
\end{enumerate}

\hypertarget{expected-output}{%
\section{Expected Output}\label{expected-output}}

\begin{verbatim}
$ ./piggie.py
usage: piggie.py [-h] [-o str] FILE [FILE ...]
piggie.py: error: the following arguments are required: FILE
$ ./piggie.py -h
usage: piggie.py [-h] [-o str] FILE [FILE ...]

Convert to Pig Latin

positional arguments:
  FILE                  Input file(s)

optional arguments:
  -h, --help            show this help message and exit
  -o str, --outdir str  Output directory (default: out-yay)
[cholla@~/work/python/playful_python/piggie]$ ./piggie.py
usage: piggie.py [-h] [-o str] FILE [FILE ...]
piggie.py: error: the following arguments are required: FILE
[cholla@~/work/python/playful_python/piggie]$ ./piggie.py -h
usage: piggie.py [-h] [-o str] FILE [FILE ...]

Convert to Pig Latin

positional arguments:
  FILE                  Input file(s)

optional arguments:
  -h, --help            show this help message and exit
  -o str, --outdir str  Output directory (default: out-yay)
$ ./piggie.py ../inputs/sonnet-29.txt
  1: sonnet-29.txt
Done, wrote 1 file to "out-yay".
$ head out-yay/sonnet-29.txt
onnet-Say 29-yay
illiam-Way akespeare-Shay

en-Whay, in-yay isgrace-day ith-way ortune-fay and-yay en-may’s-yay eyes-yay,
I-yay all-yay alone-yay eweep-bay y-may outcast-yay ate-stay,
And-yay ouble-tray eaf-day eaven-hay ith-way y-may ootless-bay ies-cray,
And-yay ook-lay upon-yay elf-mysay and-yay urse-cay y-may ate-fay,
ishing-Way e-may ike-lay o-tay one-yay ore-may ich-ray in-yay ope-hay,
eatured-Fay ike-lay im-hay, ike-lay im-hay ith-way iends-fray ossessed-pay,
esiring-Day is-thay an-may’s-yay art-yay and-yay at-thay an-may’s-yay ope-scay,
\end{verbatim}

\pagebreak

\hypertarget{solution-17}{%
\section{Solution}\label{solution-17}}

\begin{verbatim}
#!/usr/bin/env python3
"""
Author : kyclark
Date   : 2019-04-23
Purpose: Write files to Pig Latin
"""

import argparse
import os
import re
import string
import sys
from dire import warn


# --------------------------------------------------
def get_args():
    """get command-line arguments"""

    parser = argparse.ArgumentParser(
        description='Convert to Pig Latin',
        formatter_class=argparse.ArgumentDefaultsHelpFormatter)

    parser.add_argument(
        'file', metavar='FILE', nargs='+', help='Input file(s)')

    parser.add_argument(
        '-o',
        '--outdir',
        help='Output directory',
        metavar='str',
        type=str,
        default='out-yay')

    return parser.parse_args()


# --------------------------------------------------
def main():
    """Make a jazz noise here"""

    args = get_args()
    out_dir = args.outdir

    if not os.path.isdir(out_dir):
        os.makedirs(out_dir)

    num_files = 0
    for i, file in enumerate(args.file, start=1):
        basename = os.path.basename(file)
        out_file = os.path.join(out_dir, basename)
        out_fh = open(out_file, 'wt')
        print('{:3}: {}'.format(i, basename))

        if not os.path.isfile(file):
            warn('"{}" is not a file.'.format(file))
            continue

        num_files += 1
        for line in open(file):
            for bit in re.split(r"([\w']+)", line):
                out_fh.write(pig(bit))

        out_fh.close()

    print('Done, wrote {} file{} to "{}".'.format(
        num_files, '' if num_files == 1 else 's', out_dir))


# --------------------------------------------------
def pig(word):
    """Create Pig Latin version of a word"""

    if re.match(r"^[\w']+$", word):
        consonants = re.sub('[aeiouAEIOU]', '', string.ascii_letters)
        match = re.match('^([' + consonants + ']+)(.+)', word)
        if match:
            word = '-'.join([match.group(2), match.group(1) + 'ay'])
        else:
            word = word + '-yay'

    return word

# --------------------------------------------------
if __name__ == '__main__':
    main()
\end{verbatim}

\pagebreak

\hypertarget{chapter-23}{%
\section{Chapter 23}\label{chapter-23}}

\hypertarget{substring-guessing-game}{%
\section{Substring Guessing Game}\label{substring-guessing-game}}

Write a Python program called \texttt{sub.py} that plays a guessing game
where you read a \texttt{-f\textbar{}-\/-file} input (default
\texttt{/usr/share/dict/words}) and use a given
\texttt{-k\textbar{}-\/-ksize} to find all the words grouped by their
shared kmers. Remove any kmers where the number of words is fewer than
\texttt{-m\textbar{}-\/-min\_words}. Also accept a
\texttt{-s\textbar{}-\/-seed} for \texttt{random.seed} for testing
purposes. Prompt the user to guess a word for a randomly chosen kmer. If
their guess is not present in the shared list, taunt them mercilessly.
If their guess is present, affirm their worth and prompt to guess again.
Allow them to use \texttt{!} to quit and \texttt{?} to be provided a
hint (a word from the list). For both successful guesses and hints,
remove the word from the shared list. When they have quit or exhausted
the list, quit play. At the end of the game, report the number of found
words.

\pagebreak

\hypertarget{solution-18}{%
\section{Solution}\label{solution-18}}

\begin{verbatim}
#!/usr/bin/env python3
"""
Author : Ken Youens-Clark <kyclark@gmail.com>
Date   : 2019-05-08
Purpose: Find words sharing a substring
"""

import argparse
import os
import random
import re
import sys
from collections import defaultdict


# --------------------------------------------------
def get_args():
    """get command-line arguments"""
    parser = argparse.ArgumentParser(
        description='Find words sharing a substring',
        formatter_class=argparse.ArgumentDefaultsHelpFormatter)

    parser.add_argument(
        '-f',
        '--file',
        metavar='str',
        help='Input file',
        type=str,
        default='/usr/share/dict/words')

    parser.add_argument(
        '-s',
        '--seed',
        help='Random seed',
        metavar='int',
        type=int,
        default=None)

    parser.add_argument(
        '-m',
        '--min_words',
        help='Minimum number of words for a given kmer',
        metavar='int',
        type=int,
        default=3)

    parser.add_argument(
        '-k', '--ksize', help='Size of k', metavar='int', type=int, default=4)

    return parser.parse_args()


# --------------------------------------------------
def warn(msg):
    """Print a message to STDERR"""
    print(msg, file=sys.stderr)


# --------------------------------------------------
def die(msg='Something bad happened'):
    """warn() and exit with error"""
    warn(msg)
    sys.exit(1)


# --------------------------------------------------
def get_words(file):
    """Get words from input file"""

    if not os.path.isfile(file):
        die('"{}" is not a file')

    words = set()
    for line in open(file):
        for word in line.split():
            words.add(re.sub('[^a-zA-Z0-9]', '', word.lower()))

    if not words:
        die('No usable words in "{}"'.format(file))

    return words


# --------------------------------------------------
def get_kmers(words, k, min_words):
    """ Find all words sharing kmers"""

    if k <= 1:
        die('-k "{}" must be greater than 1'.format(k))

    shared = defaultdict(list)
    for word in words:
        for kmer in [word[i:i + k] for i in range(len(word) - k + 1)]:
            shared[kmer].append(word)

    # Select kmers having enough words (can't use `pop`!)

    # Method 1: for loop
    ok = dict()
    for kmer in shared:
        if len(shared[kmer]) >= min_words:
            ok[kmer] = shared[kmer]

    # Method 2: list comprehension
    # ok = dict([(kmer, shared[kmer]) for kmer in shared
    #            if len(shared[kmer]) >= min_words])

    # Method 3: map/filter
    # ok = dict(
    #     map(lambda kmer: (kmer, shared[kmer]),
    #         filter(lambda kmer: len(shared[kmer]) >= min_words,
    #                shared.keys())))

    return ok


# --------------------------------------------------
def main():
    """Make a jazz noise here"""

    args = get_args()

    random.seed(args.seed)

    shared = get_kmers(get_words(args.file), args.ksize, args.min_words)

    # Choose a kmer, setup game state
    kmer = random.choice(list(shared.keys()))
    guessed = set()
    found = []
    prompt = 'Name a word that contains "{}" [!=quit, ?=hint] '.format(kmer)
    compliments = ['Nice', 'Rock on', 'Totes', 'Fantastic', 'Excellent']
    taunts = [
        'Surely you jest!', 'Are you kidding me?',
        'You must have rocks for brains.', 'What is wrong with you?'
    ]

    #print(kmer, shared[kmer])

    while True:
        num_left = len(shared[kmer])
        if num_left == 0:
            print('No more words!')
            break

        guess = input(prompt + '({} left) '.format(num_left)).lower()

        if guess == '?':
            # Provide a hint
            pos = random.choice(range(len(shared[kmer])))
            word = shared[kmer].pop(pos)
            print('For instance, "{}"...'.format(word))

        elif guess == '!':
            # Bail
            print('Quitter!')
            break

        elif guess in guessed:
            # Chastise
            print('You have already guessed "{}"'.format(guess))

        elif guess in shared[kmer]:
            # Remove the word, feedback with compliment
            pos = shared[kmer].index(guess)
            word = shared[kmer].pop(pos)
            print('{}! "{}" is found!'.format(
                random.choice(compliments), word))
            found.append(word)
            guessed.add(guess)

        else:
            # Taunt
            print(random.choice(taunts))

    # Game over, man!
    if found:
        n = len(found)
        print('Hey, you found {} word{}! Not bad.'.format(
            n, '' if n == 1 else 's'))
    else:
        print('Wow, you found no words. You suck!')


# --------------------------------------------------
if __name__ == '__main__':
    main()
\end{verbatim}

\pagebreak

\hypertarget{chapter-24}{%
\section{Chapter 24}\label{chapter-24}}

\hypertarget{tic-tac-toe-outcome}{%
\section{Tic-Tac-Toe Outcome}\label{tic-tac-toe-outcome}}

Create a Python program called \texttt{outcome.py} that takes a given
Tic-Tac-Toe state as it's only (positional) argument and reports if X or
O has won or if there is no winner. The state should only contain the
characters ``.'', ``O'', and ``X'', and must be exactly 9 characters
long. If there is not exactly one argument, print a ``usage'' statement.

\hypertarget{expected-behavior-3}{%
\section{Expected Behavior}\label{expected-behavior-3}}

\begin{verbatim}
$ ./outcome.py
Usage: outcome.py STATE
$ ./outcome.py ..X.OA..X
State "..X.OA..X" must be 9 characters of only ., X, O
$ ./outcome.py ..X.OX...
No winner
$ ./outcome.py ..X.OX..X
X has won
\end{verbatim}

\hypertarget{test-suite}{%
\section{Test Suite}\label{test-suite}}

A passing test suite looks like this:

\begin{verbatim}
$ make test
pytest -v test.py
============================= test session starts ==============================
platform darwin -- Python 3.6.8, pytest-4.2.0, py-1.7.0, pluggy-0.8.1 -- /anaconda3/bin/python
cachedir: .pytest_cache
rootdir: /Users/kyclark/work/python/practical_python_for_data_science/ch09-python-games/exercises/tictactoe_a, inifile:
plugins: remotedata-0.3.1, openfiles-0.3.2, doctestplus-0.2.0, arraydiff-0.3
collected 3 items

test.py::test_outcome_usage PASSED                                       [ 33%]
test.py::test_outcome_bad_input PASSED                                   [ 66%]
test.py::test_outcome PASSED                                             [100%]

=========================== 3 passed in 1.20 seconds ===========================
\end{verbatim}

\pagebreak

\hypertarget{solution-19}{%
\section{Solution}\label{solution-19}}

\begin{verbatim}
#!/usr/bin/env python3
"""
Author : Ken Youens-Clark <kyclark@gmail.com>
Date   : 2019-02-06
Purpose: Rock the Casbah
"""

import os
import re
import sys


# --------------------------------------------------
def main():
    args = sys.argv[1:]

    if len(args) != 1:
        print('Usage: {} STATE'.format(os.path.basename(sys.argv[0])))
        sys.exit(1)

    state = args[0]

    if not re.search('^[.XO]{9}$', state):
        print(
            'State "{}" must be 9 characters of only ., X, O'.format(state),
            file=sys.stderr)
        sys.exit(1)

    winning = [[0, 1, 2], [3, 4, 5], [6, 7, 8], [0, 3, 6], [1, 4, 7],
               [2, 5, 8], [0, 4, 8], [2, 4, 6]]

    winner = 'No winner'

    # for player in ['X', 'O']:
    #     for combo in winning:
    #         i, j, k = combo
    #         if state[i] == player and state[j] == player and state[k] == player:
    #             winner = player
    #             break

    # for player in ['X', 'O']:
    #     for combo in winning:
    #         chars = []
    #         for i in combo:
    #             chars.append(state[i])

    #         if ''.join(chars) == player * 3:
    #             winner = player
    #             break

    # for player in ['X', 'O']:
    #     for i, j, k in winning:
    #         chars = ''.join([state[i], state[j], state[k]])
    #         if ''.join(chars) == '{}{}{}'.format(player, player, player):
    #             winner = player
    #             break

    for player in ['X', 'O']:
        for i, j, k in winning:
            combo = [state[i], state[j], state[k]]
            if combo == [player, player, player]:
                winner = '{} has won'.format(player)
                break

    # for combo in winning:
    #     group = list(map(lambda i: state[i], combo))
    #     for player in ['X', 'O']:
    #         if all(x == player for x in group):
    #             winner = player
    #             break

    print(winner)


# --------------------------------------------------
main()
\end{verbatim}

\pagebreak

\hypertarget{chapter-25}{%
\section{Chapter 25}\label{chapter-25}}

\hypertarget{twelve-days-of-christmas}{%
\section{Twelve Days of Christmas}\label{twelve-days-of-christmas}}

Write a Python program called \texttt{twelve\_days.py} that will
generate the ``Twelve Days of Christmas'' song up to the
\texttt{-n\textbar{}-\/-number\_days} argument (default \texttt{12}),
writing the resulting text to the \texttt{-o\textbar{}-\/-outfile}
argument (default STDOUT).

\begin{verbatim}
$ ./twelve_days.py -h
usage: twelve_days.py [-h] [-o str] [-n int]

Twelve Days of Christmas

optional arguments:
  -h, --help            show this help message and exit
  -o str, --outfile str
                        Outfile (STDOUT) (default: )
  -n int, --number_days int
                        Number of days to sing (default: 12)
$ ./twelve_days.py -n 1
On the first day of Christmas,
My true love gave to me,
A partridge in a pear tree.

$ ./twelve_days.py -n 3
On the first day of Christmas,
My true love gave to me,
A partridge in a pear tree.

On the second day of Christmas,
My true love gave to me,
Two turtle doves,
And a partridge in a pear tree.

On the third day of Christmas,
My true love gave to me,
Three French hens,
Two turtle doves,
And a partridge in a pear tree.

$ ./twelve_days.py -o out
$ wc -l out
     113 out
\end{verbatim}

\pagebreak

\hypertarget{solution-20}{%
\section{Solution}\label{solution-20}}

\begin{verbatim}
#!/usr/bin/env python3
"""
Author : Ken Youens-Clark <kyclark@gmail.com>
Date   : 2019-03-19
Purpose: Twelve Days of Christmas
"""

import argparse
import sys
from dire import die

# --------------------------------------------------
def get_args():
    """get command-line arguments"""
    parser = argparse.ArgumentParser(
        description='Twelve Days of Christmas',
        formatter_class=argparse.ArgumentDefaultsHelpFormatter)

    parser.add_argument('-o',
                        '--outfile',
                        help='Outfile (STDOUT)',
                        metavar='str',
                        type=str,
                        default='')

    parser.add_argument('-n',
                        '--number_days',
                        help='Number of days to sing',
                        metavar='int',
                        type=int,
                        default=12)

    return parser.parse_args()


# --------------------------------------------------
def main():
    """Make a jazz noise here"""
    args = get_args()
    out_file = args.outfile
    num_days = args.number_days

    out_fh = open(out_file, 'wt') if out_file else sys.stdout

    days = {
        12: 'Twelve drummers drumming',
        11: 'Eleven pipers piping',
        10: 'Ten lords a leaping',
        9: 'Nine ladies dancing',
        8: 'Eight maids a milking',
        7: 'Seven swans a swimming',
        6: 'Six geese a laying',
        5: 'Five gold rings',
        4: 'Four calling birds',
        3: 'Three French hens',
        2: 'Two turtle doves',
        1: 'a partridge in a pear tree',
    }

    cardinal = {
        12: 'twelfth',
        11: 'eleven',
        10: 'tenth',
        9: 'ninth',
        8: 'eighth',
        7: 'seventh',
        6: 'sixth',
        5: 'fifth',
        4: 'fourth',
        3: 'third',
        2: 'second',
        1: 'first',
    }

    if not num_days in days:
        die('Cannot sing "{}" days'.format(num_days))

    def ucfirst(s):
        return s[0].upper() + s[1:]

    for i in range(1, num_days + 1):
        first = 'On the {} day of Christmas,\nMy true love gave to me,'
        out_fh.write(first.format(cardinal[i]) + '\n')
        for j in reversed(range(1, i + 1)):
            if j == 1:
                if i == 1:
                    out_fh.write('{}.\n'.format(ucfirst(days[j])))
                else:
                    out_fh.write('And {}.\n'.format(days[j]))
            else:
                out_fh.write('{},\n'.format(days[j]))

        if i < max(days.keys()):
            out_fh.write('\n')


# --------------------------------------------------
if __name__ == '__main__':
    main()
\end{verbatim}

\pagebreak

\hypertarget{chapter-26}{%
\section{Chapter 26}\label{chapter-26}}

\hypertarget{war-card-game-in-python}{%
\section{War Card Game in Python}\label{war-card-game-in-python}}

\begin{quote}
The generation of random numbers is too important to be left to chance.
-- Robert R. Coveyou
\end{quote}

Create a Python program called ``war.py'' that plays the card game
``War.'' The program will use the \texttt{random} module to shuffle a
deck of cards, so your program will need to accept a
\texttt{-s\textbar{}-\/-seed} argument (default: \texttt{None}) which
you will use to call \texttt{random.seed}, if present.

First you program will need to create a deck of cards. You will need to
use the Unicode symbols for the suites ( ♥ ♠ ♣ ♦ ) {[}which won't
display in the PDF, so consult the Markdown file{]} and combine those
with the numbers 2-10 and the letters ``J'', ``Q,'' ``K,'' and ``A.''
(hint: look at \texttt{itertools.product}).

\begin{verbatim}
>>> from itertools import product
>>> a = list('ABC')
>>> b = range(3)
>>> list(product(a, b))
[('A', 0), ('A', 1), ('A', 2), ('B', 0), ('B', 1), ('B', 2), ('C', 0), ('C', 1), ('C', 2)]
\end{verbatim}

\textbf{NB}: You must sort your deck and then use the
\texttt{random.shuffle} method so that your cards will be in the correct
order to pass the tests!

In the real game of War, the cards are shuffled and then dealt one card
each first to the non-dealer, then to the dealer, until all cards are
dealt and each player has 26 cards. We will not be modeling this
behavior. When writing your version of the game, simply \texttt{pop} two
cards off the deck as the cards for player 1 and player 2, respectively.
Compare the two cards by ignoring the suite and evaluating the value
where 2 is the lowest and Aces are the highest. When two cards have the
same values (e.g., two 5s or two Jacks), print ``WAR!'' In the real
game, this initiates a sub-game of War which is a ``recursive''
algorithm which we will not bother modeling. Keep track of which player
wins each round where no points are awarded in a tie. At the end, report
the points for each player and state the winner. In the event of a tie,
print ``DRAW.''

\begin{verbatim}
$ ./war.py -h
usage: war.py [-h] [-s int]

"War" cardgame

optional arguments:
  -h, --help          show this help message and exit
  -s int, --seed int  Random seed (default: None)
$ ./war.py -s 1
 ♠9  ♥J P2
 ♦A  ♠5 P1
 ♣4  ♠8 P2
 ♥6  ♥3 P1
 ♥5  ♦3 P1
 ♣K ♣10 P1
 ♠7  ♦7 WAR!
 ♠2  ♦4 P2
 ♥2 ♠10 P2
 ♦6  ♣5 P1
 ♣2  ♣6 P2
 ♠4  ♥8 P2
 ♠J  ♥9 P1
♥10  ♣Q P2
 ♣8  ♥7 P1
 ♦K  ♠Q P1
♦10  ♦2 P1
 ♣9  ♦9 WAR!
 ♦8  ♣J P2
 ♣3  ♦5 P2
 ♦Q  ♥4 P1
 ♠6  ♥A P2
 ♠K  ♣7 P1
 ♥Q  ♠3 P1
 ♣A  ♥K P1
 ♠A  ♦J P1
P1 14 P2 10: Player 1 wins
$ ./war.py -s 2
 ♠4  ♠6 P2
 ♦K  ♣J P1
 ♠J  ♦4 P1
 ♣7  ♣4 P1
 ♥Q ♣10 P1
 ♦5  ♠3 P1
 ♥K  ♦9 P1
 ♥2  ♣Q P2
 ♥7  ♦A P2
 ♥3  ♥A P2
 ♣5  ♥8 P2
 ♠2 ♦10 P2
♠10  ♠K P2
 ♣2  ♦3 P2
 ♠Q  ♦8 P1
 ♦6  ♦J P2
 ♥6  ♣8 P2
 ♠8  ♦7 P1
 ♥5  ♦2 P1
 ♣6  ♥J P2
 ♥9  ♠9 WAR!
 ♣K  ♣A P2
♥10  ♦Q P2
 ♠7  ♠5 P1
 ♣9  ♠A P2
 ♥4  ♣3 P1
P1 11 P2 14: Player 2 wins
$ ./war.py -s 10
 ♥J  ♠3 P1
 ♥2  ♥5 P2
 ♦Q ♠10 P1
♣10  ♥4 P1
 ♥6  ♣5 P1
 ♦3  ♠J P2
 ♦K  ♥8 P1
 ♦5  ♣8 P2
 ♠5  ♣3 P1
 ♣J ♦10 P1
♥10  ♦J P2
 ♥A  ♣7 P1
 ♠K  ♠Q P1
 ♦7  ♠A P2
 ♣9  ♠9 WAR!
 ♣2  ♠6 P2
 ♣K  ♦A P2
 ♦6  ♥Q P2
 ♠8  ♥9 P2
 ♥3  ♠7 P2
 ♦8  ♣Q P2
 ♣6  ♠4 P1
 ♥7  ♠2 P1
 ♣4  ♦4 WAR!
 ♦9  ♦2 P1
 ♥K  ♣A P2
P1 12 P2 12: DRAW
\end{verbatim}

\pagebreak

\hypertarget{solution-21}{%
\section{Solution}\label{solution-21}}

\begin{verbatim}
#!/usr/bin/env python3
"""
Author : Ken Youens-Clark <kyclark@gmail.com>
Date   : 2019-03-18
Purpose: Cardgame "War"
"""

import argparse
import random
import sys
from itertools import product


# --------------------------------------------------
def get_args():
    """get command-line arguments"""
    parser = argparse.ArgumentParser(
        description='"War" cardgame',
        formatter_class=argparse.ArgumentDefaultsHelpFormatter)

    parser.add_argument(
        '-s',
        '--seed',
        help='Random seed',
        metavar='int',
        type=int,
        default=None)

    return parser.parse_args()


# --------------------------------------------------
def warn(msg):
    """Print a message to STDERR"""
    print(msg, file=sys.stderr)


# --------------------------------------------------
def die(msg='Something bad happened'):
    """warn() and exit with error"""
    warn(msg)
    sys.exit(1)


# --------------------------------------------------
def main():
    """Make a jazz noise here"""
    args = get_args()
    seed = args.seed

    if seed is not None:
        random.seed(seed)

    suits = list('♥♠♣♦')
    values = list(map(str, range(2, 11))) + list('JQKA')
    cards = sorted(map(lambda t: '{}{}'.format(*t), product(suits, values)))
    random.shuffle(cards)

    p1_wins = 0
    p2_wins = 0

    card_value = dict(
        list(map(lambda t: list(reversed(t)), enumerate(list(values)))))

    while cards:
        p1, p2 = cards.pop(), cards.pop()
        v1, v2 = card_value[p1[1:]], card_value[p2[1:]]
        res = ''

        if v1 > v2:
            p1_wins += 1
            res = 'P1'
        elif v2 > v1:
            p2_wins += 1
            res = 'P2'
        else:
            res = 'WAR!'

        print('{:>3} {:>3} {}'.format(p1, p2, res))

    print('P1 {} P2 {}: {}'.format(
        p1_wins, p2_wins, 'Player 1 wins'
        if p1_wins > p2_wins else 'Player 2 wins'
        if p2_wins > p1_wins else 'DRAW'))


# --------------------------------------------------
if __name__ == '__main__':
    main()
\end{verbatim}

\pagebreak

\hypertarget{chapter-27}{%
\section{Chapter 27}\label{chapter-27}}

\hypertarget{anagram}{%
\section{Anagram}\label{anagram}}

Write a program called \texttt{presto.py} that will find anagrams of a
given positional argument. The program should take an optional
\texttt{-w\textbar{}-\/-wordlist} (default
\texttt{/usr/share/dict/words}) and produce output that includes
combinations of \texttt{-n\textbar{}num\_combos} words (default
\texttt{1}) that are anagrams of the given input.

\begin{verbatim}
$ ./presto.py
usage: presto.py [-h] [-w str] [-n int] [-d] str
presto.py: error: the following arguments are required: str
$ ./presto.py -h
usage: presto.py [-h] [-w str] [-n int] [-d] str

Find anagrams

positional arguments:
  str                   Input text

optional arguments:
  -h, --help            show this help message and exit
  -w str, --wordlist str
                        Wordlist (default: /usr/share/dict/words)
  -n int, --num_combos int
                        Number of words combination to test (default: 1)
  -d, --debug           Debug (default: False)
$ ./presto.py presto
presto =
   1. poster
   2. repost
   3. respot
   4. stoper
$ ./presto.py listen
listen =
   1. enlist
   2. silent
   3. tinsel
$ ./presto.py listen -n 2 | tail
  82. sten li
  83. te nils
  84. ten lis
  85. ten sil
  86. ti lens
  87. til ens
  88. til sen
  89. tin els
  90. tin les
  91. tinsel
\end{verbatim}

\pagebreak

\hypertarget{solution-22}{%
\section{Solution}\label{solution-22}}

\begin{verbatim}
#!/usr/bin/env python3
"""
Author : Ken Youens-Clark <kyclark@gmail.com>
Date   : 2019-05-19
Purpose: Find anagrams
"""

import argparse
import logging
import os
import re
import sys
from collections import defaultdict, Counter
from itertools import combinations, permutations, product, chain
from dire import warn, die


# --------------------------------------------------
def get_args():
    """get command-line arguments"""
    parser = argparse.ArgumentParser(
        description='Find anagrams',
        formatter_class=argparse.ArgumentDefaultsHelpFormatter)

    parser.add_argument('text', metavar='str', help='Input text')

    parser.add_argument('-w',
                        '--wordlist',
                        help='Wordlist',
                        metavar='str',
                        type=str,
                        default='/usr/share/dict/words')

    parser.add_argument('-n',
                        '--num_combos',
                        help='Number of words combination to test',
                        metavar='int',
                        type=int,
                        default=1)

    parser.add_argument('-d', '--debug', help='Debug', action='store_true')

    return parser.parse_args()


# --------------------------------------------------
def main():
    """Make a jazz noise here"""
    args = get_args()
    text = args.text
    word_list = args.wordlist

    if not os.path.isfile(word_list):
        die('--wordlist "{}" is not a file'.format(word_list))

    logging.basicConfig(
        filename='.log',
        filemode='w',
        level=logging.DEBUG if args.debug else logging.CRITICAL)

    words = defaultdict(set)
    for line in open(word_list):
        for word in line.split():
            clean = re.sub('[^a-z0-9]', '', word.lower())
            if len(clean) == 1 and clean not in 'ai':
                continue
            words[len(clean)].add(clean)

    text_len = len(text)
    counts = Counter(text)
    anagrams = set()
    lengths = list(words.keys())
    for i in range(1, args.num_combos + 1):
        key_combos = list(
            filter(
                lambda t: sum(t) == text_len,
                set(
                    map(lambda t: tuple(sorted(t)),
                        combinations(chain(lengths, lengths), i)))))

        for keys in key_combos:
            logging.debug('Searching keys {}'.format(keys))
            word_combos = list(product(*list(map(lambda k: words[k], keys))))

            for t in word_combos:
                if Counter(''.join(t)) == counts:
                    for p in filter(
                            lambda x: x != text,
                            map(lambda x: ' '.join(x), permutations(t))):
                        anagrams.add(p)

            logging.debug('# anagrams = {}'.format(len(anagrams)))

    logging.debug('Finished searching')

    if anagrams:
        print('{} ='.format(text))
        for i, t in enumerate(sorted(anagrams), 1):
            print('{:4}. {}'.format(i, t))
    else:
        print('No anagrams for "{}".'.format(text))


# --------------------------------------------------
if __name__ == '__main__':
    main()
\end{verbatim}

\pagebreak

\hypertarget{chapter-28}{%
\section{Chapter 28}\label{chapter-28}}

\hypertarget{hangman}{%
\section{Hangman}\label{hangman}}

Write a Python program called \texttt{hangman.py} that will play a game
of Hangman which is a bit like ``Wheel of Fortune'' where you present
the user with a number of elements indicating the length of a word. For
our game, use the underscore \texttt{\_} to indicate a letter that has
not been guessed. The program should take
\texttt{-n\textbar{}-\/-minlen} minimum length (default \texttt{5}) and
\texttt{-l\textbar{}-\/-maxlen} maximum length options (default
\texttt{10}) to indicate the minimum and maximum lengths of the randomly
chosen word taken from the \texttt{-w\textbar{}-\/-wordlist} option
(default \texttt{/usr/share/dict/words}). It also needs to take
\texttt{-s\textbar{}-\/-seed} to for the random seed and the
\texttt{-m\textbar{}-\/-misses} number of misses to allow the player.

To play, you will initiate an inifinite loop and keep track of the game
state, e.g., the word to guess, the letters already guessed, the letters
found, the number of misses. As this is an interactive game, I cannot
write an test suite, so you can play my version and then try to write
one like it. If the user guesses a letter that is in the word, replace
the \texttt{\_} characters with the letter. If the user guesses the same
letter twice, admonish them. If the user guesses a letter that is not in
the word, increment the misses and let them know they missed. If the
user guesses too many times, exit the game and insult them. If they
correctly guess the word, let them know and exit the game.

\begin{verbatim}
$ ./hangman.py -h
usage: hangman.py [-h] [-l MAXLEN] [-n MINLEN] [-m MISSES] [-s SEED]
                  [-w WORDLIST]

Hangman

optional arguments:
  -h, --help            show this help message and exit
  -l MAXLEN, --maxlen MAXLEN
                        Max word length (default: 10)
  -n MINLEN, --minlen MINLEN
                        Min word length (default: 5)
  -m MISSES, --misses MISSES
                        Max number of misses (default: 10)
  -s SEED, --seed SEED  Random seed (default: None)
  -w WORDLIST, --wordlist WORDLIST
                        Word list (default: /usr/share/dict/words)
$ ./hangman.py
_ _ _ _ _ _ _ _ (Misses: 0)
Your guess? ("?" for hint, "!" to quit) a
_ _ _ _ _ _ _ _ (Misses: 1)
Your guess? ("?" for hint, "!" to quit) i
_ _ _ _ _ _ i _ (Misses: 1)
Your guess? ("?" for hint, "!" to quit) e
_ _ _ _ _ _ i _ (Misses: 2)
Your guess? ("?" for hint, "!" to quit) o
_ o _ _ _ _ i _ (Misses: 2)
Your guess? ("?" for hint, "!" to quit) u
_ o _ _ _ _ i _ (Misses: 3)
Your guess? ("?" for hint, "!" to quit) y
_ o _ _ _ _ i _ (Misses: 4)
Your guess? ("?" for hint, "!" to quit) c
_ o _ _ _ _ i _ (Misses: 5)
Your guess? ("?" for hint, "!" to quit) d
_ o _ _ _ _ i _ (Misses: 6)
Your guess? ("?" for hint, "!" to quit) p
_ o _ _ _ _ i p (Misses: 6)
Your guess? ("?" for hint, "!" to quit) m
_ o _ _ _ _ i p (Misses: 7)
Your guess? ("?" for hint, "!" to quit) n
_ o _ _ _ _ i p (Misses: 8)
Your guess? ("?" for hint, "!" to quit) s
_ o s _ s _ i p (Misses: 8)
Your guess? ("?" for hint, "!" to quit) t
_ o s t s _ i p (Misses: 8)
Your guess? ("?" for hint, "!" to quit) h
You win. You guessed "hostship" with "8" misses!
$ ./hangman.py -m 2
_ _ _ _ _ _ _ _ _ _ (Misses: 0)
Your guess? ("?" for hint, "!" to quit) a
_ _ _ _ _ _ a _ _ a (Misses: 0)
Your guess? ("?" for hint, "!" to quit) b
_ _ _ _ _ _ a _ _ a (Misses: 1)
Your guess? ("?" for hint, "!" to quit) c
You lose, loser!  The word was "metromania."
\end{verbatim}

\pagebreak

\hypertarget{solution-23}{%
\section{Solution}\label{solution-23}}

\begin{verbatim}
#!/usr/bin/env python3
"""
Author:  Ken Youens-Clark <kyclark@gmail.com>
Purpose: Hangman game
"""

import argparse
import os
import random
import re
import sys
from dire import die


# --------------------------------------------------
def get_args():
    """parse arguments"""
    parser = argparse.ArgumentParser(
        description='Hangman',
        formatter_class=argparse.ArgumentDefaultsHelpFormatter)

    parser.add_argument('-l',
                        '--maxlen',
                        help='Max word length',
                        type=int,
                        default=10)

    parser.add_argument('-n',
                        '--minlen',
                        help='Min word length',
                        type=int,
                        default=5)

    parser.add_argument('-m',
                        '--misses',
                        help='Max number of misses',
                        type=int,
                        default=10)

    parser.add_argument('-s',
                        '--seed',
                        help='Random seed',
                        type=str,
                        default=None)

    parser.add_argument('-w',
                        '--wordlist',
                        help='Word list',
                        type=str,
                        default='/usr/share/dict/words')

    return parser.parse_args()


# --------------------------------------------------
def bail(msg):
    """Print a message to STDOUT and quit with no error"""
    print(msg)
    sys.exit(0)


# --------------------------------------------------
def main():
    """main"""
    args = get_args()
    max_len = args.maxlen
    min_len = args.minlen
    max_misses = args.misses
    wordlist = args.wordlist

    random.seed(args.seed)

    if not os.path.isfile(wordlist):
        die('--wordlist "{}" is not a file.'.format(wordlist))

    if min_len < 1:
        die('--minlen must be positive')

    if not 3 <= max_len <= 20:
        die('--maxlen should be between 3 and 20')

    if min_len > max_len:
        die('--minlen ({}) is greater than --maxlen ({})'.format(
            min_len, max_len))

    good_word = re.compile('^[a-z]{' + str(min_len) + ',' + str(max_len) +
                           '}$')
    words = [w for w in open(wordlist).read().split() if good_word.match(w)]

    word = random.choice(words)
    play({'word': word, 'max_misses': max_misses})


# --------------------------------------------------
def play(state):
    """Loop to play the game"""
    word = state.get('word') or ''

    if not word: die('No word!')

    guessed = state.get('guessed') or list('_' * len(word))
    prev_guesses = state.get('prev_guesses') or set()
    num_misses = state.get('num_misses') or 0
    max_misses = state.get('max_misses') or 0

    if ''.join(guessed) == word:
        msg = 'You win. You guessed "{}" with "{}" miss{}!'
        bail(msg.format(word, num_misses, '' if num_misses == 1 else 'es'))

    if num_misses >= max_misses:
        bail('You lose, loser!  The word was "{}."'.format(word))

    print('{} (Misses: {})'.format(' '.join(guessed), num_misses))
    new_guess = input('Your guess? ("?" for hint, "!" to quit) ').lower()

    if new_guess == '!':
        bail('Better luck next time, loser.')
    elif new_guess == '?':
        new_guess = random.choice([x for x in word if x not in guessed])
        num_misses += 1

    if not re.match('^[a-z]$', new_guess):
        print('"{}" is not a letter'.format(new_guess))
        num_misses += 1
    elif new_guess in prev_guesses:
        print('You already guessed that')
    elif new_guess in word:
        prev_guesses.add(new_guess)
        last_pos = 0
        while True:
            pos = word.find(new_guess, last_pos)
            if pos < 0:
                break
            elif pos >= 0:
                guessed[pos] = new_guess
                last_pos = pos + 1
    else:
        num_misses += 1

    play({
        'word': word,
        'guessed': guessed,
        'num_misses': num_misses,
        'prev_guesses': prev_guesses,
        'max_misses': max_misses
    })


# --------------------------------------------------
if __name__ == '__main__':
    main()
\end{verbatim}

\pagebreak

\hypertarget{chapter-29}{%
\section{Chapter 29}\label{chapter-29}}

\hypertarget{markov-chain}{%
\section{Markov Chain}\label{markov-chain}}

Write a Python program called \texttt{markov.py} that takes one or more
text files as positional arguments for training. Use the
\texttt{-n\textbar{}-\/-num\_words} argument (default \texttt{2}) to
find clusters of words and the words that follow them, e.g., in ``The
Bustle'' by Emily Dickinson:

\begin{verbatim}
The bustle in a house
The morning after death
Is solemnest of industries
Enacted upon earth,—

The sweeping up the heart,
And putting love away
We shall not want to use again
Until eternity.
\end{verbatim}

If \texttt{n=1}, then we find that ``The'' can be followed by
``bustle,'' ``morning,'' and ``sweeping. There is a''the" followed by
``heart,'' but we're not going to alter the text in any way, including
removing punctuation, so just use \texttt{str.split} on the text to
break up the words.

To begin your text, choose a random word (or words) that begin with an
uppercase letter. Then randomly select the next word in the chain, keep
track of the floating window of the \texttt{-n} words, and keep
selecting the next words until you have matched or exceeded the
\texttt{-l\textbar{}-\/-length} argument of the number of characters
(default 500) to emit at which point you should stop when you find a
word that terminates with \texttt{.}, \texttt{!}, or \texttt{?}.

If you use \texttt{str.split} to get the words from the training text,
you'll be removing any newlines from the text, so use a
\texttt{-w\textbar{}-\/-text\_width} argument (default 70) to introduce
newlines in the output before the text exceeds that number of characters
on the line.

Because of the use of randomness, you should include a
\texttt{-s\textbar{}-\/-seed} argument (default \texttt{None}) to pass
to \texttt{random.seed}.

Occassionally you may chose a path that terminates. That is, in
selecting the next word, you may find there is no next-next word. In
that case, just exit the program.

My implementation includes a \texttt{-d\textbar{}-\/-debug} option that
will write a \texttt{.log} file so you can inspect my data structures
and logic as you write your own version.

You should find many diverse texts and use them all as training files
with varying numbers for \texttt{-n} to see how the texts will be mixed.
The results are endlessly entertaining.

\begin{verbatim}
$ ./markov.py
usage: markov.py [-h] [-l int] [-n int] [-s int] [-w int] [-d] FILE [FILE ...]
markov.py: error: the following arguments are required: FILE
$ ./markov.py -h
usage: markov.py [-h] [-l int] [-n int] [-s int] [-w int] [-d] FILE [FILE ...]

Markov Chain

positional arguments:
  FILE                  Training file(s)

optional arguments:
  -h, --help            show this help message and exit
  -l int, --length int  Output length (characters) (default: 500)
  -n int, --num_words int
                        Number of words (default: 2)
  -s int, --seed int    Random seed (default: None)
  -w int, --text_width int
                        Max number of characters per line (default: 70)
  -d, --debug           Debug to ".log" (default: False)
$ ./markov.py ../inputs/const.txt
Discoveries; To constitute Tribunals inferior to the seat of the
Senate and House of Representatives shall have been committed, which
district shall have the Qualifications requisite for Electors of the
sixth Year, so that one third may be imposed on such Importation, not
exceeding three on the Journal. Neither House, during the Time of
Adjournment, he may require it. No Bill of Attainder or ex post facto
Law shall be established by Law: but the Party convicted shall
nevertheless be liable and subject to their Consideration such
Measures as he shall nominate, and by and with the Advice and Consent
of the government of the United States under this Constitution, or,
on the List the said Office, the same State claiming Lands under
Grants of different States; between Citizens of each shall constitute
a Quorum to do Business; but a smaller number may adjourn from day to
day, and may be included within this Union, according to their
Consideration such Measures as he shall nominate, and by and with the
Advice and Consent of the United States.
\end{verbatim}

\pagebreak

\hypertarget{solution-24}{%
\section{Solution}\label{solution-24}}

\begin{verbatim}
#!/usr/bin/env python3
"""
Author : Ken Youens-Clark <kyclark@gmail.com>
Date   : 2019-05-23
Purpose: Markov Chain
"""

import argparse
import logging
import os
import random
import string
import sys
from pprint import pprint as pp
from collections import defaultdict


# --------------------------------------------------
def get_args():
    """Get command-line arguments"""

    parser = argparse.ArgumentParser(
        description='Markov Chain',
        formatter_class=argparse.ArgumentDefaultsHelpFormatter)

    parser.add_argument('training',
                        metavar='FILE',
                        nargs='+',
                        type=argparse.FileType('r'),
                        help='Training file(s)')

    parser.add_argument('-l',
                        '--length',
                        help='Output length (characters)',
                        metavar='int',
                        type=int,
                        default=500)

    parser.add_argument('-n',
                        '--num_words',
                        help='Number of words',
                        metavar='int',
                        type=int,
                        default=2)

    parser.add_argument('-s',
                        '--seed',
                        help='Random seed',
                        metavar='int',
                        type=int,
                        default=None)

    parser.add_argument('-w',
                        '--text_width',
                        help='Max number of characters per line',
                        metavar='int',
                        type=int,
                        default=70)

    parser.add_argument('-d',
                        '--debug',
                        help='Debug to ".log"',
                        action='store_true')

    return parser.parse_args()


# --------------------------------------------------
def main():
    """Make a jazz noise here"""

    args = get_args()
    num_words = args.num_words
    char_max = args.length
    text_width = args.text_width

    random.seed(args.seed)

    logging.basicConfig(
        filename='.log',
        filemode='w',
        level=logging.DEBUG if args.debug else logging.CRITICAL)

    all_words = defaultdict(list)
    for fh in args.training:
        words = fh.read().split()

        for i in range(0, len(words) - num_words):
            l = words[i:i + num_words + 1]
            all_words[tuple(l[:-1])].append(l[-1])

    logging.debug('all words = {}'.format(all_words))

    prev = ''
    while not prev:
        start = random.choice(
            list(
                filter(lambda w: w[0][0] in string.ascii_uppercase,
                       all_words.keys())))
        if all_words[start]:
            prev = start

    logging.debug('Starting with "{}"'.format(prev))

    p = ' '.join(prev)
    char_count = len(p)
    print(p, end=' ')
    line_width = char_count

    while True:
        if not prev in all_words: break

        new_word = random.choice(all_words[prev])
        new_len = len(new_word) + 1
        logging.debug('chose = "{}" from {}'.format(new_word, all_words[prev]))

        if line_width + new_len > text_width:
            print()
            line_width = new_len
        else:
            line_width += new_len

        char_count += new_len
        print(new_word, end=' ')
        if char_count >= char_max and new_word[-1] in '.!?': break
        prev = prev[1:] + (new_word, )

    logging.debug('Finished')
    print()


# --------------------------------------------------
if __name__ == '__main__':
    main()
\end{verbatim}

\pagebreak

\hypertarget{chapter-30}{%
\section{Chapter 30}\label{chapter-30}}

\hypertarget{morse-encoderdecoder}{%
\section{Morse Encoder/Decoder}\label{morse-encoderdecoder}}

Write a Python program called \texttt{morse.py} that will
encrypt/decrypt text to/from Morse code. The program should expect a
single positional argument which is either the name of a file to read
for the input or the character \texttt{-} to indicate reading from
STDIN. The program should also take a \texttt{-c\textbar{}-\/-coding}
option to indicate use of the \texttt{itu} or standard \texttt{morse}
tables, \texttt{-o\textbar{}-\/-outfile} for writing the output (default
STDOUT), and a \texttt{-d\textbar{}-\/-decode} flag to indicate that the
action is to decode the input (the default is to encode it).

\begin{verbatim}
$ ./morse.py
usage: morse.py [-h] [-c str] [-o str] [-d] [-D] FILE
morse.py: error: the following arguments are required: FILE
$ ./morse.py -h
usage: morse.py [-h] [-c str] [-o str] [-d] [-D] FILE

Encode and decode text/Morse

positional arguments:
  FILE                  Input file or "-" for stdin

optional arguments:
  -h, --help            show this help message and exit
  -c str, --coding str  Coding version (default: itu)
  -o str, --outfile str
                        Output file (default: None)
  -d, --decode          Decode message from Morse to text (default: False)
  -D, --debug           Debug (default: False)
$ ./morse.py ../inputs/fox.txt
- .... .  --.- ..- .. -.-. -.-  -... .-. --- .-- -.  ..-. --- -..-  .--- ..- -- .--. ...  --- ...- . .-.  - .... .  .-.. .- --.. -.--  -.. --- --. .-.-.-
[cholla@~/work/python/playful_python/morse]$ ./morse.py ../inputs/fox.txt | ./morse.py -d -
THE QUICK BROWN FOX JUMPS OVER THE LAZY DOG.
\end{verbatim}

\pagebreak

\hypertarget{solution-25}{%
\section{Solution}\label{solution-25}}

\begin{verbatim}
#!/usr/bin/env python3
"""
Author :    Ken Youens-Clark <kyclark@gmail.com>
Date   :    2019-04-29
Purpose:    Morse en/decoder
Background: https://en.wikipedia.org/wiki/Morse_code,
            https://en.wikipedia.org/wiki/American_Morse_code

"""

import argparse
import logging
import random
import re
import string
import sys


# --------------------------------------------------
def get_args():
    """Get command-line arguments"""

    parser = argparse.ArgumentParser(
        description='Encode and decode text/Morse',
        formatter_class=argparse.ArgumentDefaultsHelpFormatter)

    parser.add_argument('input',
                        metavar='FILE',
                        help='Input file or "-" for stdin')

    parser.add_argument('-c',
                        '--coding',
                        help='Coding version',
                        metavar='str',
                        type=str,
                        choices=['itu', 'morse'],
                        default='itu')

    parser.add_argument('-o',
                        '--outfile',
                        help='Output file',
                        metavar='str',
                        type=str,
                        default=None)

    parser.add_argument('-d',
                        '--decode',
                        help='Decode message from Morse to text',
                        action='store_true')

    parser.add_argument('-D', '--debug', help='Debug', action='store_true')

    return parser.parse_args()


# --------------------------------------------------
def encode_word(word, table):
    """Encode word using given table"""

    coded = []
    for char in word.upper():
        logging.debug(char)
        if char != ' ' and char in table:
            coded.append(table[char])

    encoded = ' '.join(coded)
    logging.debug('endoding "{}" to "{}"'.format(word, encoded))

    return encoded


# --------------------------------------------------
def decode_word(encoded, table):
    """Decode word using given table"""

    decoded = []
    for code in encoded.split(' '):
        if code in table:
            decoded.append(table[code])

    word = ''.join(decoded)
    logging.debug('dedoding "{}" to "{}"'.format(encoded, word))

    return word


# --------------------------------------------------
def test_encode_word():
    """Test Encoding"""

    assert encode_word('sos', ENCODE_ITU) == '... --- ...'
    assert encode_word('sos', ENCODE_MORSE) == '... .,. ...'


# --------------------------------------------------
def test_decode_word():
    """Test Decoding"""

    assert decode_word('... --- ...', DECODE_ITU) == 'SOS'
    assert decode_word('... .,. ...', DECODE_MORSE) == 'SOS'


# --------------------------------------------------
def test_roundtrip():
    """Test En/decoding"""

    random_str = lambda: ''.join(random.sample(string.ascii_lowercase, k=10))
    for _ in range(10):
        word = random_str()
        for encode_tbl, decode_tbl in [(ENCODE_ITU, DECODE_ITU),
                                       (ENCODE_MORSE, DECODE_MORSE)]:

            assert word.upper() == decode_word(encode_word(word, encode_tbl),
                                               decode_tbl)


# --------------------------------------------------
def main():
    """Make a jazz noise here"""
    args = get_args()
    action = 'decode' if args.decode else 'encode'
    output = open(args.outfile, 'wt') if args.outfile else sys.stdout
    source = sys.stdin if args.input == '-' else open(args.input)

    coding_table = ''
    if args.coding == 'itu':
        coding_table = ENCODE_ITU if action == 'encode' else DECODE_ITU
    else:
        coding_table = ENCODE_MORSE if action == 'encode' else DECODE_MORSE

    logging.basicConfig(
        filename='.log',
        filemode='w',
        level=logging.DEBUG if args.debug else logging.CRITICAL)

    word_split = r'\s+' if action == 'encode' else r'\s{2}'

    for line in source:
        for word in re.split(word_split, line):
            if action == 'encode':
                print(encode_word(word, coding_table), end='  ')
            else:
                print(decode_word(word, coding_table), end=' ')
        print()


# --------------------------------------------------
def invert_dict(d):
    """Invert a dictionary's key/value"""

    #return dict(map(lambda t: list(reversed(t)), d.items()))
    return dict([(v, k) for k, v in d.items()])


# --------------------------------------------------
# GLOBALS

ENCODE_ITU = {
    'A': '.-',
    'B': '-...',
    'C': '-.-.',
    'D': '-..',
    'E': '.',
    'F': '..-.',
    'G': '--.',
    'H': '....',
    'I': '..',
    'J': '.---',
    'K': '-.-',
    'L': '.-..',
    'M': '--',
    'N': '-.',
    'O': '---',
    'P': '.--.',
    'Q': '--.-',
    'R': '.-.',
    'S': '...',
    'T': '-',
    'U': '..-',
    'V': '...-',
    'W': '.--',
    'X': '-..-',
    'Y': '-.--',
    'Z': '--..',
    '0': '-----',
    '1': '.----',
    '2': '..---',
    '3': '...--',
    '4': '....-',
    '5': '.....',
    '6': '-....',
    '7': '--...',
    '8': '---..',
    '9': '----.',
    '.': '.-.-.-',
    ',': '--..--',
    '?': '..--..',
    '!': '-.-.--',
    '&': '.-...',
    ';': '-.-.-.',
    ':': '---...',
    "'": '.----.',
    '/': '-..-.',
    '-': '-....-',
    '(': '-.--.',
    ')': '-.--.-',
}

ENCODE_MORSE = {
    'A': '.-',
    'B': '-...',
    'C': '..,.',
    'D': '-..',
    'E': '.',
    'F': '.-.',
    'G': '--.',
    'H': '....',
    'I': '..',
    'J': '-.-.',
    'K': '-.-',
    'L': '+',
    'M': '--',
    'N': '-.',
    'O': '.,.',
    'P': '.....',
    'Q': '..-.',
    'R': '.,..',
    'S': '...',
    'T': '-',
    'U': '..-',
    'V': '...-',
    'W': '.--',
    'X': '.-..',
    'Y': '..,..',
    'Z': '...,.',
    '0': '+++++',
    '1': '.--.',
    '2': '..-..',
    '3': '...-.',
    '4': '....-',
    '5': '---',
    '6': '......',
    '7': '--..',
    '8': '-....',
    '9': '-..-',
    '.': '..--..',
    ',': '.-.-',
    '?': '-..-.',
    '!': '---.',
    '&': '.,...',
    ';': '...,..',
    ':': '-.-,.,.',
    "'": '..-.,.-..',
    '/': '..-,-',
    '-': '....,.-..',
    '(': '.....,.-..',
    ')': '.....,..,..',
}

DECODE_ITU = invert_dict(ENCODE_ITU)
DECODE_MORSE = invert_dict(ENCODE_MORSE)

# --------------------------------------------------
if __name__ == '__main__':
    main()
\end{verbatim}

\pagebreak

\hypertarget{chapter-31}{%
\section{Chapter 31}\label{chapter-31}}

\hypertarget{rot13-rotate-13}{%
\section{ROT13 (Rotate 13)}\label{rot13-rotate-13}}

Write a Python program called \texttt{rot13.py} that will
encrypt/decrypt input text by shifting the text by a given
\texttt{-s\textbar{}-\/-shift} argument or will move each character
halfway through the alphabet, e.g., ``a'' becomes ``n,'' ``b'' becomes
``o,'' etc. The text to rotate should be provided as a single positional
argument to your program and can either be a text file, text on the
command line, or \texttt{-} to indicate STDIN so that you can round-trip
data through your program to ensure you are encrypting and decrypting
properly.

\hypertarget{discussion-1} operator:

\begin{verbatim}
>>> now = 8
>>> (now + 6) % 12
2
\end{verbatim}

And 6 hours from 8AM is, indeed, 2PM.

Similarly if I want to know how many hours (in decimal) are a particular
number of minutes, I need to mod by 60:

\begin{verbatim}
>>> minutes = 90
>>> int(minutes / 60) + (minutes % 60) / 60
1.5
>>> minutes = 204
>>> int(minutes / 60) + (minutes % 60) / 60
3.4
\end{verbatim}

If you \texttt{import\ string}, you can see all the lower/uppercase
letters

\begin{verbatim}
>>> import string
>>> string.ascii_lowercase
'abcdefghijklmnopqrstuvwxyz'
>>> string.ascii_uppercase
'ABCDEFGHIJKLMNOPQRSTUVWXYZ'
\end{verbatim}

So I think about ``rot13'' like adding 13 (or some other shift interval)
to the position of the letter in the list and modding by the length of
the list to wrap it around. If the shift is 13 and we are at ``a'' and
want to know what the letter 13 way is, we can use \texttt{pos} to find
``a'' and add 13 to that:

\begin{verbatim}
>>> lcase = list(string.ascii_lowercase)
>>> lcase.index('a')
0
>>> lcase[lcase.index('a') + 13]
'n'
\end{verbatim}

But if we want to know the value for something after the 13th letter in
our list, we are in trouble!

\begin{verbatim}
>>> lcase[lcase.index('x') + 13]
Traceback (most recent call last):
  File "<stdin>", line 1, in <module>
IndexError: list index out of range
\end{verbatim}

\texttt{\%} to the rescue!

\begin{verbatim}
>>> lcase[(lcase.index('x') + 13) % len(lcase)]
'k'
\end{verbatim}

It's not necessary in this algorithm to shift by any particular number.
13 is special because it's halfway through the alphabet, but we could
shift by just 2 or 5 characters. If we want to round-trip our text, it's
necessary to shift in the opposite direction on the second half of the
trip, so be sure to use the negative value there!

\hypertarget{expected-behavior-4}{%
\section{Expected Behavior}\label{expected-behavior-4}}

\begin{verbatim}
$ ./rot13.py
usage: rot13.py [-h] [-s int] str
rot13.py: error: the following arguments are required: str
$ ./rot13.py -h
usage: rot13.py [-h] [-s int] str

Argparse Python script

positional arguments:
  str                  Input text, file, or "-" for STDIN

optional arguments:
  -h, --help           show this help message and exit
  -s int, --shift int  Shift arg (default: 0)
$ ./rot13.py AbCd
NoPq
$ ./rot13.py AbCd -s 2
CdEf
$ ./rot13.py fox.txt
Gur dhvpx oebja sbk whzcf bire gur ynml qbt.
$ ./rot13.py fox.txt | ./rot13.py -
The quick brown fox jumps over the lazy dog.
$ ./rot13.py -s 3 fox.txt | ./rot13.py -s -3 -
The quick brown fox jumps over the lazy dog.
\end{verbatim}

\pagebreak

\hypertarget{solution-26}{%
\section{Solution}\label{solution-26}}

\begin{verbatim}
#!/usr/bin/env python3
"""
Author : Ken Youens-Clark <kyclark@gmail.com>
Date   : 2019-05-09
Purpose: Rot13 (rotate 13) encryption
"""

import argparse
import os
import re
import string
import sys


# --------------------------------------------------
def get_args():
    """get command-line arguments"""
    parser = argparse.ArgumentParser(
        description='ROT13 encryption',
        formatter_class=argparse.ArgumentDefaultsHelpFormatter)

    parser.add_argument('text',
                        metavar='str',
                        help='Input text, file, or "-" for STDIN')

    parser.add_argument('-s',
                        '--shift',
                        help='Shift arg',
                        metavar='int',
                        type=int,
                        default=0)

    return parser.parse_args()


# --------------------------------------------------
def main():
    """Make a jazz noise here"""
    args = get_args()
    text = args.text

    if text == '-':
        text = sys.stdin.read()
    elif os.path.isfile(text):
        text = open(text).read()

    lcase = list(string.ascii_lowercase)
    ucase = list(string.ascii_uppercase)
    num_lcase = len(lcase)
    num_ucase = len(ucase)
    lcase_shift = args.shift or int(num_lcase / 2)
    ucase_shift = args.shift or int(num_ucase / 2)

    def rot13(char):
        if char in lcase:
            pos = lcase.index(char)
            rot = (pos + lcase_shift) % num_lcase
            return lcase[rot]
        elif char in ucase:
            pos = ucase.index(char)
            rot = (pos + ucase_shift) % num_ucase
            return ucase[rot]
        else:
            return char

    print(''.join(map(rot13, text)).rstrip())


# --------------------------------------------------
if __name__ == '__main__':
    main()
\end{verbatim}

\pagebreak

\end{document}
